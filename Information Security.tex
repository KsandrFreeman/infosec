\documentclass[10pt,a4paper]{book}
%\documentclass[12pt,report,russian]{ncc}
%\usepackage{a4wide}
\usepackage{cmap}                       % Поддержка поиска русских слов в PDF (pdflatex)
\usepackage[T1, T2A]{fontenc}
\usepackage[utf8]{inputenc}
\usepackage[english,russian]{babel}
%\usepackage{pscyr}                      % поддержка русских type1 шрифтов из пакета PSCyr
\usepackage{indentfirst}                % Красная строка в первом абзаце
%\usepackage{misccorr}
%Может быть установлено 8pt, 9pt, 10pt, 11pt, 12pt, 14pt, 17pt, and 20pt
%\usepackage[12pt]{extsizes}
%\usepackage[mag=1000,a4paper,left=3cm,right=2cm,top=2cm,bottom=2cm,noheadfoot]{geometry}

\usepackage{amsfonts, amsmath, eucal, bm, amssymb, graphicx, color}
\usepackage{algorithm, algorithmic}     % 'algorithm' environments
\usepackage{multirow}                   % multirow cells in tables
\usepackage{arydshln}                   % dash lines in tables
\usepackage{subfig, float, wrapfig}     % sub figures
\usepackage{makeidx, caption}           % index, titles for figures
\usepackage{enumerate}
\usepackage{fancybox}                   % страница в рамке
%\usepackage{fancyhdr}                   % глава и секция вверху страницы
%\usepackage{layout}
\usepackage[left=1.84cm, right=1.5cm, paperwidth=14cm, top=1.8cm, bottom=2cm, height=19.8cm, paperheight=20cm]{geometry}

% поддержка гиперссылок; гиперссылки в pdf, должен быть последним загруженным пакетом
\ifx\pdfoutput\undefined
    \usepackage[unicode,dvips]{hyperref}
\else
    \usepackage[pdftex,colorlinks,unicode,bookmarks]{hyperref}
\fi

%\paperwidth=16.8cm \oddsidemargin=0cm \evensidemargin=0cm \hoffset=-0.33cm \textwidth=13.2cm
%\paperheight=24cm \voffset=-0.4cm \topmargin=0cm \headsep=0cm \headheight=0cm \textheight=19.8cm \footskip=0.9cm

% параметры PDF файла
\hypersetup{
    pdftitle={Защита информации},
    pdfauthor={Э. М. Габидулин, А. С. Кшевецкий, А. И. Колыбельников},
    pdfsubject=учебное пособие,
    pdfkeywords={защита информации, криптография, МФТИ}
}

% добавить точку после номера секции, раздела и т.д.
\makeatletter
\def\@seccntformat#1{\csname the#1\endcsname.\quad}
\def\numberline#1{\hb@xt@\@tempdima{#1\if&#1&\else.\fi\hfil}}
\makeatother

% перенос слов с тире
%\lccode`\-=`\-
%\defaulthyphenchar=127

% изменить подписи к рисункам, таблицам и т.д.
\captionsetup{labelsep=period}          % заменить : на .
\captionsetup{textformat=period}        % Подписи завершать точкой
%\captionsetup[table]{position=above}    % вертикальные отступы подписи таблицы для случая, когда подпись вверху
%\captionsetup[figure]{position=below}   % вертикальные отступы подписи рисунка для случая, когда подпись внизу

%% стиль главы и секции вверху страницы
%\pagestyle{fancy}
%%\renewcommand{\chaptermark}[1]{\markboth{#1}{}}
%\renewcommand{\sectionmark}[1]{\markright{#1}{}}
%
%%\fancyhf{}
%%\fancyfoot[СE,CO]{\thepage}
%%\fancyhead[LE]{\textsc{\nouppercase{\leftmark}}}
%\fancyhead[RO]{\textsc{\nouppercase{\rightmark}}}
%
%\fancypagestyle{plain}{ %
%\fancyhf{}                              % remove everything
%\renewcommand{\headrulewidth}{0pt}      % remove lines as well
%\renewcommand{\footrulewidth}{0pt}}

% запретить выходить за границы страницы
\sloppy

\newtheorem{theorem}{Теорема}[section]
\newtheorem{lemma}[theorem]{Лемма}
\newtheorem{definition}[theorem]{Определение}
\newtheorem{property}[theorem]{Утверждение}
\newtheorem{corollary}[theorem]{Следствие}
%\newtheorem{algorithm}[theorem]{Алгоритм}
\newtheorem{remark}[theorem]{Замечание}
\newcommand{\proof}{\noindent\textsc{Доказательство.\ }}

%\newtheorem{example}{\textsc{\textbf{Пример}}}
\newcommand{\example}{\textsc{\textbf{Пример.}} }
\newcommand{\exampleend}

\newcommand{\set}[1]{\mathbb{#1}}
\newcommand{\group}[1]{\mathbb{#1}}
\newcommand{\E}{\group{E}}
\newcommand{\F}{\group{F}}
\newcommand{\GF}[1]{\group{GF}(#1)}
\newcommand{\Gr}{\group{G}}
\newcommand{\R}{\group{R}}
\newcommand{\Z}{\group{Z}}
\newcommand{\MAC}{\textrm{MAC}}
\newcommand{\HMAC}{\textrm{HMAC}}
\newcommand{\PK}{\textrm{PK}}
\newcommand{\SK}{\textrm{SK}}

%Наконец, существует способ дублировать знаки операций, который мы приведем безо всяких пояснений. Включив
%\newcommand*{\hm}[1]{#1\nobreak\discretionary{}{\hbox{\mathsurround=0pt #1}}{}}
%в преамбулу, можно написать $a\hm+b\hm+c\hm+d$, при этом в формуле a\hm+b\hm+c\hm+d при переносе знак + будет продублирован.

% Дублирование символов бинарных операций ("+", "-", "="), набранных в строчных формулах, при переносе на другую строку:
%%begin{latexonly}
%\renewcommand\ne{\mathchar"3236\mathchar"303D\nobreak
%      \discretionary{}{\usefont
%      {OMS}{cmsy}{m}{n}\char"36\usefont
%      {OT1}{cmr}{m}{n}\char"3D}{}}
%\begingroup
%\catcode`\+\active\gdef+{\mathchar8235\nobreak\discretionary{}%
% {\usefont{OT1}{cmr}{m}{n}\char43}{}}
%\catcode`\-\active\gdef-{\mathchar8704\nobreak\discretionary{}%
% {\usefont{OMS}{cmsy}{m}{n}\char0}{}}
%\catcode`\=\active\gdef={\mathchar12349\nobreak\discretionary{}%
% {\usefont{OT1}{cmr}{m}{n}\char61}{}}
%\endgroup
%\def\cdot{\mathchar8705\nobreak\discretionary{}%
% {\usefont{OMS}{cmsy}{m}{n}\char1}{}}
%\def\times{\mathchar8706\nobreak\discretionary{}%
% {\usefont{OMS}{cmsy}{m}{n}\char2}{}}
%\mathcode`\==32768
%\mathcode`\+=32768
%\mathcode`\-=32768
%%end{latexonly}

\makeindex

\begin{document}

%\layout

% рамка границ страницы http://www.ctan.org/tex-archive/macros/latex/contrib/fancybox/fancybox-doc.pdf
% сделать поиск по fancypage, thisfancypage
%\thisfancypage{}{\fbox}
%\thisfancypage{\fbox}{}
%\fancypage{}{\fbox}         % закомментировать
%\fancypage{\fbox}{\fbox}    % закомментировать
%\fancypage{\setlength{\fboxsep}{32pt}\fbox}{}

\title{Защита информации \\ Учебное пособие}
\author{Габидулин Эрнст Мухамедович \\ Кшевецкий Александр Сергеевич \\ Колыбельников Александр Иванович}
\date{
 %   \textbf{\textsc{Черновой вариант. Может содержать ошибки.}} \\
%    \today
}
\maketitle
\setcounter{page}{3}

\newpage
%\thispagestyle{empty}
\setcounter{tocdepth}{2}
\tableofcontents
%\thispagestyle{empty}
\newpage

%\lhead[\leftmark]{}
%\rhead[]{\rightmark}

\input{foreword}

\chapter{Основные понятия и определения}

\section{Краткая история криптографии}

Вслед за возникновением письменности появилась задача обеспечения секретности и подлинности передаваемых сообщений путем так называемой тайнописи. Поскольку государства возникали почти одновременно с письменностью, дипломатия и военное управление требовали секретности.

Данные о первых способах тайнописи весьма обрывочны. Предполагается, что тайнопись была известна в древнем Египте и Вавилоне. До нашего времени дошли литературные свидетельства того, что  секретное письмо использовалось в древней Греции. Наиболее известен метод шифрования, который использовался Гаем Юлием Цезарем (100--44 гг. до н.э.).

Первое известное исследование по анализу стойкости методов шифрования было сделано в <<Манускрипте о дешифровании криптографических сообщений>> Абу Аль-Кинди (801–-873 гг. н.э.). Он показал, что моноалфавитные шифры, в которых каждому символу кодируемого текста ставится в однозначное соответствие  какой-то другой символ алфавита, легко поддаются частотному криптоанализу. Абу Аль-Кинди был так же знаком с более сложными полиалфавитными шифрами.

В европейских странах полиалфавитные шифры были открыты в эпоху Возрождения. Итальянский архитектор Баттиста Альберти (1404--1472) изобрел полиалфавитный шифр, который, впоследствии, получил имя дипломата XVI века Блеза де Виженера. В истории развития полиалфавитных шифров до XX века, также, наиболее известны немецкий аббат XVI века Иоганн Трисемус и английский ученый начала XIX века Чарльз Витстон. Витстон изобрел простой и стойкий способ полиалфавитной замены, называемый шифром Плейфера по имени лорда Плейфера, способствовавшему внедрению шифра. Шифр Плейфера использовался вплоть до Первой мировой войны.

Прообразом современных шифров для электронно-вычислительных машин стали так называемые роторные машины XX века, которые позволяли создавать и реализовывать устойчивые к взлому полиалфавитные шифры. Примером такой машины является немецкая машина \emph{Enigma}\index{Enigma}, разработанная в конце первой мировой войны. Период активного применения Enigma пришелся на Вторую мировую войну.

Появление в середине  XX столетия первых ЭВМ кардинально изменило ситуацию. Вычислительные способности компьютеров подняли на совершенно новый уровень как возможности реализации шифров, недоступных ранее из-за их высокой сложности, так и возможности криптоаналитиков по их взлому. Следствием этого факта стало разделение шифров по области применения.

В 1976 году появился шифр DES (Data Encryption Standard), который был принят как стандарт США. DES широко использовался для шифрования пакетов данных при передаче в компьютерных сетях и системах хранения данных. С 90-х годов параллельно с традиционными шифрами, основой которых была булева алгебра,  активно развиваются шифры, основанные на операциях в конечном поле.  Широкое распространение персональных компьютеров и быстрый рост объема передаваемых данных в компьютерных сетях,  привело к замене в 2002 году стандарта DES  на более стойкий и быстрый в программной реализации стандарт -- шифр AES (Advanced Encryption Standard). Окончательно, DES был выведен из эксплуатации как стандарт в 2005 году.

В беспроводных голосовых сетях передачи данных используются шифры с малой задержкой шифрования и расшифрования на основе посимвольных преобразований -- так называемые \emph{потоковые шифры}.

%Основным их преимуществом является сочетание помехоустойчивого кодирования с криптостойкостью шифра.

Параллельно с разработкой быстрых шифров, в 1977 г. появился новый класс криптосистем, так называемые \emph{криптосистемы с открытым ключом}. Хотя эти новые криптосистемы намного медленнее (технически сложнее) симметричных, они открыли принципиально новые возможности --  \emph{цифровая подпись}, \emph{аутентификация} и \emph{сертификация} составили основу современной защищенной связи в Интернете.

В настоящее время типичное использование криптографии в информационных системах состоит в:
\begin{itemize}
\item цифровой аутентификации пользователей с помощью криптосистем с открытым ключом,
\item создании кратковременных сеансовых ключей и
\item применении быстрых шифров в процессах обмена данными.
\end{itemize}


\input{model_of_the_transmission_system_with_crypto}

\section{Классификация криптосистем и шифров}

%\subsection{Секретные и открытые ключи}

%Различают два класса криптосистем -- с секретным и с открытым ключом.

%\sub
\input{secret-key_cryptosystem}

\input{public-key_cryptosystems}

\subsection{Шифры замены и перестановки}

По способу преобразования открытого текста в шифрованный текст шифры разделяются на шифры замены и шифры перестановки.

\input{substitution_ciphers}

\input{permutation_ciphers}

\input{composite_chiphers}

\subsection{Примеры современных криптопримитивов}

Приведем примеры названий некоторых современных криптографических примитивов, из которых строят системы защиты информации:
\begin{itemize}
    \item DES, AES, ГОСТ 28147-89, Blowfish, RC5, RC6 -- блоковые симметричные шифры, скорость обработки десятки мегабайт в секунду,
    \item A5/1, A5/2, A5/3, RC4 -- потоковые симметричные шифры с высокой скоростью, семейство A5 применяется в мобильной связи GSM, RC4 -- в компьютерных сетях для SSL соединения между браузером и вебсервером,
    \item RSA -- криптосистема с открытым ключом для шифрования,
    \item RSA, DSA, ГОСТ Р 34.10-2001 -- криптосистемы с открытым ключом для электронно-цифровой подписи,
    \item MD5, SHA-1, SHA-2, ГОСТ Р 34.11-94 -- криптографические хэш-функции.
\end{itemize}

\input{Cryptanalysis_methods_and_types_of_attacks}

\input{The_minimum_key_lengths}

\chapter{Классические шифры}

В главе приведены наиболее известные \emph{классические} шифры,  которыми можно было пользоваться до появления роторных машин. К ним относятся шифр Цезаря, шифр Плейфера--Витстона, шифр Хилла, шифр Виженера. Они очень наглядно демонстрируют различные классы шифров.

\section{Моноалфавитные шифры}
\selectlanguage{russian}


Преобразования открытого текста в шифротекст могут быть описаны различными функциями. Если функция преобразования является аддитивной, то и соответствующий шифр называется \textbf{аддитивным}. Если это преобразование является аффинным, то шифр называется \textbf{аффинным}.

\subsection{Шифр Цезаря}

Известным примером простого  шифра замены является \textbf{шифр Цезаря}\index{шифр!Цезаря}.  Процедура шифрования поясняется с помощью рисунка
%\ref{fig:caesar}
приведенного ниже и состоит в следующем. Записывают все буквы латинского алфавита в стандартном порядке
    \[ A B C D E \dots Z. \]
Делают циклический сдвиг вправо, например, на три буквы и записывают все буквы во втором ряду, начиная с третьей буквы $C$. Буквы первого ряда заменяют соответствующими (как показано стрелкой на рисунке) буквами второго ряда. После такой замены слова не распознаются теми, кто не знает ключа. Ключом $K$ является первый символ сдвинутого алфавита.

\[ \begin{array}{ccccccccc}
    \text{A} & \text{B} & \text{C} & \text{D} & \text{E} & & \text{X} & \text{Y} & \text{Z} \\
    \downarrow & \downarrow & \downarrow & \downarrow & \downarrow & \dots & \downarrow & \downarrow & \downarrow \\
    \text{C} & \text{D} & \text{E} & \text{F} & \text{G} & & \text{Z} & \text{A} & \text{B} \\
\end{array} \]

\example
В русском языке сообщение \textsc{изучайтекриптографию} посредством шифрования с ключом $K = \text{\textsc{в}}$ (сдвиг вправо на 2 символа по алфавиту) преобразуется в \textsc{лкцъгмхивнултсжугчла}.
\exampleend

Недостатком любого шифра замены является то, что в шифрованном тексте сохраняются все частоты появления букв открытого текста и корреляционные связи между буквами. Они существуют в каждом языке. Например, в русском языке чаще всего встречаются буквы $A$ и $O$. Для дешифрования криптоаналитик имеет возможность прочитать открытый текст, используя частотный анализ букв шифротекста. Для "взлома" шифра Цезаря достаточно найти одну пару букв -- одну замену.


\subsection{Аддитивный шифр перестановки}

Следующий рисунок
%Рисунок \ref{fig:zamena}
поясняет \textbf{аддитивный шифр} перестановки\index{шифр!аддитивный перестановки} на алфавите. Все 26 букв латинского алфавита нумеруют по порядку от 0 до 25. Затем номер буквы меняют в соответствии с уравнением
    \[ y = x + b \mod 26, \]
где $x$ -- прежний номер, $y$ -- новый номер, $b$ -- заданное целое число, определяющее сдвиг номера и известное только легальным пользователям. Очевидно, что шифр Цезаря является примером аддитивного шифра.
  	
\[ \begin{array}{ccccccccc}
    \text{A} & \text{B} & \text{C} & \text{D} & \text{E} & & \text{X} & \text{Y} & \text{Z} \\
    \downarrow & \downarrow & \downarrow & \downarrow & \downarrow & \dots & \downarrow & \downarrow & \downarrow \\
    2 & 3 & 4 & 5 & 6 & & 25 & 0 & 1 \\
\end{array} \]


\subsection{Аффинный шифр}

Аддитивный шифр является частным случаем \textbf{аффинного шифра}\index{шифр!афинный}. Правило шифрования сообщения имеет вид
    \[ y = a x + b \mod n. \]
Здесь производится умножение номера символа $x$ из алфавита, $x\in \set\{ 0, 1, 2, \dots, N \leq n-1 \}$,  на заданное целое число $a$ и сложение с числом $b$ по модулю целого числа $n$. Ключом является $K = (a, b)$.

Расшифрование осуществляется по формуле
    \[ x = (y - b) a^{-1} \mod n. \]

Чтобы обеспечить обратимость в этом шифре, должен существовать единственный обратный элемент $a^{-1}$ по модулю $n$. Для этого должно выполняться условие $\gcd(a,n) = 1$, то есть $a$ и $n$ должны быть взаимно простыми числами ($\gcd$ -- обозначение термина с английского  great common divider -- наибольший общий делитель, $\text{НОД}$). Очевидно, что для "взлома" такого шифра достаточно найти две пары букв -- две замены.


\input{bigrammnye_substitution_ciphers}

\input{hills_cipher}

% \subsection{Омофонные замены}
%
% Омофонными заменами называют криптопримитивы, в основе которых лежит замена групп символов открытого текста $M$ на группу символов $C$ с использованием ключа $K$. Такой метод шифрования вносит неоднозначность между $M$ и $C$, это позволяет защититься от методов  частотного криптоанализа.
%  \subsection{шифрокоды}
%  Шифрокоды - это класс шифров сочетающих в себе свойства кодов и помехозащищенности со свойствами шифра и обеспечения конфиденциальности.

\input{vigeneres_chipher}

\input{polyalphabetic_cipher_cryptanalysis}

\input{perfect_secure_systems}

\chapter{Блоковые шифры}

\input{Feistel_cipher}

\input{GOST_28147-89}

\input{AES}

\input{Block_cipher_modes}

\section{Некоторые свойства блоковых шифров}

\input{feistel_network_reversibility}

\input{Feistel_cipher_without_s_blocks}

\input{Avalanche_effect}

\input{double_and_triple_ciphering}

\chapter{Потоковые шифры}

\input{per-char_encryption}

\section{Криптостойкие последовательности} %{Генерирование криптостойких псевдослучайных последовательностей бит}

Генераторы псевдослучайных чисел (ГПСЧ) ставят в соответствие набору символов $z_{1} z_{2} \dots z_{l}$ значение некоторой функции $f(z_{1} z_{2} \dots z_{l}) = f_{1}$. Следующим $l$ символам -- $f(z_{l+1} z_{l+2} \dots z_{2l}) = f_{2}$. Получают набор значений $f_{1} f_{2} \dots f_{N}$ и используют их в качестве случайной последовательности.

Для проверки степени близости к независимости и равномерному распределению символов существует набор тестов. Американский институт стандартизации NIST разработал 16 тестов на псевдослучайность. Подсчитывается число нулей и единиц, число одинаковых соседних пар, число одинаковых подпоследовательностей, автокорреляция, частота следующего символа в зависимости от предыдущих и т.д. Вычисляют вероятность символа 0 (или 1)
\[
    P(f_i = 0 | f_{i-1} f_{i-2} \dots f_{i-k}) = \frac{1}{2} - \epsilon,
\]
Если вычисления осуществляются за полиномиальное время от длины подпоследовательности $k$, то есть с количеством битовых операций $O(k^{\textrm{const}})$, то тест называют \emph{полиномиальным}\index{задача!полиномиальная}.

Псевдослучайная последовательность удовлетворяет \textbf{тесту <<следующего бита>>}, если не существует полиномиального по $k$ теста, позволяющего по предыдущим $k$ битам определить следующий бит с вероятностью отличной от $\frac{1}{2} - \epsilon$, принимая во внимание погрешность оценки $\epsilon$. Последовательность, удовлетворяющая тесту <<следующего бита>>, также удовлетворяет всем возможным полиномиальным тестам по $k$ на равномерность распределения, и наоборот.

Последовательность называется \emph{криптографически стойкой} или \emph{криптостойкой}, если она удовлетворяет тесту <<следующего бита>>.

\input{bbs_generator}

%\input{micalis_generator}

\section{Последовательности максимального периода}
\selectlanguage{russian}
\label{section:max-seq}

Периодические последовательности с максимальным периодом называются $M$-последовательностями\index{$M$-последовательность} и характеризуются идеальной функцией автокорреляции.

Пусть $M$-последовательность имеет вид $x_1 x_2 \dots x_n$, где значения $x_{i} = \pm 1$. Функция автокорреляции\index{функция!автокорреляции} такой последовательности имеет вид
\[
    \sum\limits_{i=1}^{T} x_i x_{i + \tau}  = \left\{ \begin{array}{l}
        -1, ~ \tau \neq 0, \\
        T, ~ \tau = 0. \\
    \end{array} \right.
\]

$M$-последовательности можно генерировать с помощью регистров сдвига с линейной обратной связью. На рис. \ref{fig:impulse} показан регистр сдвига, состоящий из ячеек памяти со входами для информационных символов и тактовых импульсов. На рис. \ref{fig:generator} тактовые импульсы опущены. Через $S_{j}$ обозначено содержимое $j$-й ячейки, где $j=\overline{0,L-1}$. В цепь обратной связи введены сумматоры по модулю 2 и умножители с коэффициентами $c_{j}$, принимающими значения $0,1$.  Поступление тактового импульса вызывает сдвиг в регистре сдвига.
\begin{figure}[h!]
	\centering
    \includegraphics[width=0.8\textwidth]{pic/taktovi-impuls}
    \caption{Тактовые импульсы\label{fig:impulse}}
\end{figure}

В этой схеме используется линейная обратная связь. Выход ячейки с номером $L-1$ умножают на $c_{1}$,  выход следующей ячейки $L-2$ умножают на $c_2$ и так далее. После умножения выходы ячеек суммируют по модулю 2 и результат подают на вход крайней ячейки $(L-1)$. Если содержимое всех ячеек состоит из нулей, то  генерируемая последовательность  также состоит из одних нулей. Если имеется ненулевое заполнение $S_{j-1}, S_{j-2}, \dots, S_{j-L}$, то после поступления $j$-го тактового импульса имеем сигнал на выходе такой, как показано  на рис. \ref{fig:generator}:
\begin{figure}[h!]
	\centering
	\includegraphics[width=0.8\textwidth]{pic/generator}
    \caption{Генератор\label{fig:generator}}
\end{figure}

Во всем разделе \ref{section:max-seq}  операцией <<$+$>>  обозначается операция сложения двоичных коэффициентов и многочленов по модулю 2.
\[
    S_{j} = c_{1} S_{j-1} + c_{2} S_{j-2} +  \dots  + c_{L-1} S_{j-L+1} + c_{L} S_{j-L}.
\]

Это соотношение определяет принцип  работы генератора на регистрах сдвига\index{регистр сдвига}. Всего $2^{L}$ начальных условий, задающих значения бит в ячейках, из них $2^{L}-1$ ненулевых начальных условий. Генерируемая двоичная последовательность является периодической с периодом $T\leq 2^{L}-1$. Величина  периода зависит от коэффициентов обратной связи $c_{1},  \ldots,  c_{L} $.

Набор коэффициентов задается многочленом обратной связи
    \[ c(y) = 1 + c_1 y+ c_2 y^2 + \dots + c_L y^L. \]
Свойства этого многочлена влияют на период генерируемой последовательности. Рассмотрим его подробнее.

Многочлен $c(y)$ над полем $\GF{2}$ с коэффициентами $c_i \in \GF{2}$ называется \textit{приводимым}, если его можно представить в виде произведения многочленов меньшей степени. Например, многочлен $1 + y^{2} = (1 + y) (1 + y) = 1 + y + y + y^2 = 1 + y^2$ является приводимым. Многочлен $1 + y + y^{2}$ -- неприводимый.

Приведем без доказательства два важных утверждения.
\begin{itemize}
    \item Пусть $c(y)$ -- неприводимый многочлен. Тогда существует такое значение $m$, что $y^{m} + 1$ делится без остатка на $c(y)$, то есть $\frac{y^{m} + 1}{c(y)} = d(y)$.
    \item Существуют многочлены $c(y)$, для которых $m=2^{L} - 1$, где L -- степень многочлена $c(y)$. Эти многочлены называются примитивными.
\end{itemize}

Если $c(y)$ -- примитивный многочлен\index{многочлен!примитивный}, то период генерируемой последовательности является максимальным, то есть равным $T = 2^{L} - 1$. Генерируемые последовательности являются $M$-последовательностями, то есть последовательностями максимального периода. Для любого начального ненулевого состояния генерируется циклический сдвиг одной и той же последовательности максимального периода $T=2^{L} - 1$.

Оказывается, что многочлен обратной связи и состояние регистра определяются однозначно по $2L$ последовательным символам выхода регистра сдвига с линейной обратной связью (с помощью алгоритма Берлекэмпа-Мэсси или алгоритма Евклида).

Например, в спутниках GPS\index{GPS} имеется регистр сдвига c 43 ячейками и периодом генерируемой последовательности $2^{43} - 1$. Длительность одного импульса $\sim 0,1$ мкс, период последовательности примерно равен одному году. Если бы для генерирования криптостойкой последовательности был просто применен регистр сдвига с линейной обратной связью, то, чтобы найти многочлен обратной связи, криптоаналитику достаточно было получить 86 символов последовательности.

Генератор псевдослучайной последовательности максимального периода, построенный на регистре сдвига с линейной обратной связью, нельзя считать криптостойким из-за возможности по $2L$ последовательным битам восстановить многочлен обратной связи. Поэтому с криптографической точки зрения требует решения задача улучшения последовательности, используемой для шифрования.


\section[Три способа улучшения последовательностей]{Три способа улучшения \protect\\ последовательностей}

\input{generators_with_multile_shift_registers}

\input{generators_with_nonlinear_transformations}

\input{majority_generators}

\chapter{Криптографические хэш-функции}

\input{hash-function_properties}

\input{GOST_R_34.11-94.tex}

\section{Коды аутентификации сообщений}
\selectlanguage{russian}

Для обеспечения целостности и подтверждения авторства информации, передаваемой по каналу связи, используют \textbf{коды аутентификации сообщений}, $\MAC$ (message authentication code).

Кодом аутентификации сообщения называется \emph{криптографическая хэш-функция} $\MAC(K,m)$, зависящая от передаваемого сообщения $m$ и секретного ключа $K$ отправителя $A$, обладающая свойствами цифровой подписи:
\begin{itemize}
    \item получатель $B$, используя такой же или другой ключ, имеет возможность проверить целостность и доказать принадлежность информации $A$;
    \item код аутентификации невозможно фальсифицировать.
\end{itemize}

Код аутентификации может быть построен либо на симметричной криптосистеме, в таком случае обе стороны имеют один общий секретный ключ, либо на криптосистеме с открытым ключом, в которой $A$ использует свой секретный ключ, а $B$ -- открытый ключ отправителя $A$.

Наиболее универсальный способ аутентификации сообщений через схемы ЭЦП на криптосистемах с открытым ключом состоит в том, что сторона $A$ отправляет стороне $B$ сообщение
    \[ m ~\|~ \textrm{ЭЦП}(K, h(m)), \]
где $h(m)$ -- криптографическая хэш-функция в схеме ЭЦП.  Для аутентификации большого объема информации этот способ не подходит из-за медленной операции вычисления подписи. Например, вычисление одной ЭЦП на криптосистемах с открытым ключом занимает порядка 10 мс на ПК. При средней длине IP-пакета 1 Кб, для каждого из которых требуется вычислить код аутентификации, получим максимальную пропускную способность в $\frac{1 ~ \text{Kб}}{10 ~ \text{мс}} = 100$ Кб/с.

Поэтому для большого объема данных, которые нужно аутентифицировать, $A$ и $B$ создают общий секретный ключ аутентификации $K$. Далее код аутентификации вычисляется либо с помощью модификации блокового шифра, либо с помощью криптографической хэш-функции.

Для каждого пакета информации $m$ отправитель $A$ вычисляет $\MAC(K,m)$ и присоединяет его к сообщению $m$:
    \[ m ~ \|~ \MAC(K,m). \]
 Зная секретный ключ $K$, получатель $B$ может удостовериться с помощью кода аутентификации, что информация не была изменена, фальсифицирована, а была создана отправителем.

Требования к длине кода аутентификации в общем случае  такие же, как и для криптографической хэш-функции, то есть длина должна быть не менее 160--256 бит. На практике часто используют усеченный код аутентификации. Например, в IPsec код аутентификации IP-пакета занимает 96 бит для уменьшения избыточности, что ведет к снижению криптостойкости.

Стандартные способы использования кода аутентификации сообщения следующие.
\begin{itemize}
    \item Если шифрование данных не применяется, отправитель $A$ для каждого пакета информации $m$ отсылает сообщение
        \[ m ~\|~ \MAC(K, m) .\]
    \item Если используется шифрование данных симметричной криптосистемой с помощью ключа $K_e$, то код аутентификации с ключом $K_a$ может вычисляться как до, так и после шифрования:
        \[ E_{K_e}(m) ~\|~ \MAC(K_a, E_{K_e}(m)) ~~ \text{ или } ~~ E_{K_e}(m ~\|~ \MAC(K_a, m)). \]

\end{itemize}
Первый способ, используемый в IPsec, хорош тем, что для проверки целостности достаточно вычислить только код аутентификации, тогда как во втором случае нужно дополнительно вначале расшифровать данные. С другой стороны, во втором способе, используемом в системе PGP, защищенность кода аутентификации не зависит от потенциальной уязвимости алгоритма шифрования.

Вычисление кода аутентификации от пакета информации $m$ с использованием блокового шифра $E$ осуществляется в виде
    \[ \MAC(K, m) = E_K(H(m)), \]
где $H$ -- криптографическая хэш-функция.

Код аутентификации на основе хэш-функции обозначается $\HMAC$ (Hash-based MAC)\index{HMAC} и стандартно вычисляется в виде
    \[ \HMAC(K, m) = H(K \| H(K \| m)), \]
где $\|$ является операцией конкатенации битовых строк. Возможно также вычисление в виде
    \[ \HMAC(K, m) = H(K \| m \| K)). \]

В протоколе IPsec (часть протокола IPv6) используется следующее вычисление кода аутентификации:
    \[ \HMAC(K, m) = H((K \oplus ~ \textrm{opad}) ~\|~ H((K \oplus ~ \textrm{ipad}) ~\|~ m)), \]
где $\textrm{opad}$ -- последовательность повторяющихся байтов
    \[ \text{\texttt{0x5C}}= [1011100]_2, \]
$\textrm{ipad}$ -- последовательность повторяющихся байтов
    \[ \text{\texttt{0x36}} = [00110110]_2, \]
которые инвертируют половину бит ключа. Считается, что использование различных значений ключа повышает криптостойкость.

В протоколе защищенной связи SSL/TLS, используемом в Интернете для инкапсуляции протокола HTTP в протокол SSL (HTTPS), код $\HMAC$ определяется почти так же, как в IPsec. Отличие состоит в том, что вместо операции XOR для последовательностей $\textrm{ipad}$ и $\textrm{opad}$ осуществляется конкатенация:
    \[ \HMAC(K, m) = H((K ~\|~ \textrm{opad}) ~\|~ H((K ~\|~ \textrm{ipad}) ~\|~ m)). \]

Двойное хэширование\index{двойное хэширование} с ключом в
    \[ \HMAC(K, m) = H(K \| H(K \| m)) \]
применяется для защиты от атаки на расширение сообщений. Вычисление хэш-функции от сообщения $m$, состоящего из $n$ блоков $m_1 m_2 \dots m_n$, можно записать в виде
    \[ H_i = f(H_{i-1}, m_i), ~ H_0 \equiv IV = \textrm{const}, ~ H(m) \equiv H_n, \]
где $f$ -- известная сжимающая функция.

Пусть код аутентификации использует одинарное хэширование с ключом:
    \[ \MAC(K, m) = H(K \| m) = H (m_0 = K \| m_1 \| m_2 \| \dots \| m_n). \]
Тогда криптоаналитик, не зная секретного ключа, имеет возможность вычислить код аутентификации для некоторого расширенного сообщения $m \| m_{n+1}$:
\[
    \MAC(K, m \| m_{n+1}) = \underbrace{H \left( K \| m_1 \| m_2 \| \dots \| m_n \right.}_{\MAC(K, m)} \left. \| m_{n+1} \right) =
\] \[
    f(\MAC(K, m), m_{n+1}).
\]


\section{Коллизии в хэш-функциях}

\input{birthdays_paradox}

\input{collisions_probability}

\input{hash-functions_combinations}

\chapter{Криптосистемы с открытым ключом}

\textbf{Криптосистемой с открытым ключом} (public-key cryptosystem, PKC) называется криптографическое преобразование, использующее  два ключа -- открытый и секретный. Пара из \textbf{секретного}\index{ключ!секретный} (private key, secret key, SK) и \textbf{открытого}\index{ключ!открытый} (public key, PK) ключей создается пользователем, который свой секретный ключ держит в секрете, а открытый ключ делает общедоступным для всех пользователей. Криптографическое преобразование в одну сторону (шифрование) можно выполнить  зная только открытый ключ, а в другую (расшифрование) -- только зная секретный ключ. Во многих криптосистемах из секретного ключа можно вычислить открытый ключ.

Если прямое преобразование выполняется открытым ключом, а обратное -- секретным, то криптосистема называется схемой \textbf{шифрования с открытым ключом}. Все пользователи, зная открытый ключ получателя, могут зашифровать для него сообщение, которое может расшифровать только владелец секретного ключа.

Если прямое преобразование выполняется секретным ключом, а обратное -- открытым, то криптосистема называется схемой \textbf{электронно-цифровой подписи, ЭЦП}, -- владелец секретного ключа может \emph{подписать} (зашифровать) сообщение, а все пользователи, зная открытый ключ, могут проверить, что подпись (шифротекст) была создана только владельцем секретного ключа и никем другим.

Некоторые криптосистемы с открытым ключом могут использоваться и как схема шифрования, и как схема ЭЦП с одним алгоритмом, например криптосистема RSA. В общем случае это не так. Например, есть криптосистема с открытым ключом Эль-Гамаля и другой алгоритм -- схема ЭЦП Эль-Гамаля.

Для более простого опубликования открытых ключей производители операционных систем, браузеров и программных продуктов часто встраивают наборы открытых ключей различных организаций. Браузеры и операционные системы содержат десятки-сотни встроенных открытых ключей, принадлежащих центрам выдачи сертификатов X509, производителям программным продуктов, банковским и интернет-сервисам. Фактически все пользователи \textbf{доверяют}\index{доверие} производителям ПО и тому, что открытые ключи, встроенные в продукт, действительно принадлежат их истинным владельцам.

Криптосистемы с открытым ключом построены на основе односторонних (однонаправленных) функций c потайным входом. Под \textbf{односторонней} функцией понимают \emph{вычислительную} невозможность вычисления ее обращения: вычисление значения функции $y = f(x)$ при заданном аргументе $x$ является легкой задачей, вычисление аргумента $x$ при заданном значении функции $y$ -- трудной задачей.

Рассмотрим, например, умножение больших целых чисел $p,q$, каждое из которых состоит из 500 бит в двоичной записи. Как известно, вычисление их произведения является легкой задачей, сложность которой, по порядку, не превышает $500^2$ битовых операций. Однако, обратная задача -- разложить 1000-битовое число $n=pq$, представляющее собой произведение двух простых 500-битовых чисел, на простые множители является трудной задачей, связанной с поиском делителей, сложность которой оценивается экспонентой от корня квадратного длины числа ($ \exp(1000^1/2)$).

Односторонняя функция $y = f(x,K)$ с \textbf{потайным входом}\index{функция!с потайным входом} $K$ определяется как  функция, которая легко вычисляется при заданном $x$, и аргумент $x$ которой можно легко вычислить из $y$, если известен "секретный" параметр $K$ , и вычислить невозможно, если параметр $K$ неизвестен.

Криптосистемы с открытым ключом построены на дискретной математике. Необходимые математические основы модульной арифметики, групп, полей и простых чисел приведены в Приложении \ref{chap:discrete-math}.

\input{rsa}

\input{el-gamal}

\input{GOST_R_34.10-2001.tex}

\input{pki_key_length}

\chapter{Распространение ключей}

Задачей распространения ключей между двумя пользователями является создание секретных псевдослучайных сеансовых ключей шифрования и аутентификации сообщений. Пользователи предварительно создают и обмениваются ключами аутентификации один раз. В дальнейшем для создания защищенной связи пользователи производят взаимную аутентификацию и вырабатывают сеансовые ключи\index{ключ!сеансовый}.

\input{shamirs_three-step_protocol}

\section{Протоколы с симметричными шифрами}

\subsection{Аутентификация и атаки воспроизведения}

Рассмотрим такую ситуацию: обе стороны $A$ и $B$ имеют общий долговременный ключ $K_{AB}$ и симметричную систему шифрования. Нужно выработать сеансовый секретный ключ $K$. Сторона $A$ создает ключ $K$ и желает его передать стороне $B$.

\begin{enumerate}
    \item Для этого сторона $A$ с помощью общего ключа $K_{AB}$ создает и передает $B$ шифрованное сообщение:
            \[ A \rightarrow B: ~ E_{K_{AB}}(K, B, A). \]
        В этом сообщении имеются так называемые поля -- $(B,A)$ -- информация для дополнительного подтверждения.
    \item Сторона $B$, используя общий ключ $K_{AB}$, расшифровывает полученное сообщение:
            \[ D_{K_{AB}}( E_{K_{AB}}( K, B, A)) = (K, B, A). \]
        В результате  сторона $B$ получает сеансовый ключ $K$ и дополнительные данные $(B,A)$.
\end{enumerate}

Недостаток этого протокола состоит в том, что криптоаналитик может перехватывать сообщения и через некоторое время переслать их стороне $A$.

Рассмотрим другие варианты решения задачи о передаче сеансового ключа.
Задача остается прежней: обе стороны $A$ и $B$ имеют общий долговременный секретный ключ $K_{AB}$, сторона $A$ должна  выработать сеансовый секретный ключ $K$ и доставить стороне $B$.

Протокол включает \textbf{метки времени} -- информацию о моменте $t_A$ отправки сообщения и моменте получения сообщения $t_B$.

\begin{enumerate}
    \item  Сторона $A$ вырабатывает $K$ и с помощью долговременного ключа $K_{AB}$ создает шифрованное сообщение с меткой времени $t_A$ и передает его стороне $B$:
            \[ A \rightarrow B: ~ E_{K_{AB}}(K, t_A). \]
    \item Сторона $B$ получает сообщение и расшифровывает его с помощью общего ключа:
            \[ D_{K_{AB}}( E_{K_{AB}}( K, t_A) = (K, t_A). \]
        В результате $B$ получает $(K, t_A)$, то есть, секретный ключ и метку времени. $B$ измеряет время прихода $t_B$ и интервал запаздывания. Если $|t_B - t_A| \le \delta$, то $B$ аутентифицирует $A$.
\end{enumerate}
Метка времени является одноразовой меткой и защищает от атак воспроизведения ранее записанных сообщений.

Рассмотрим другой способ передачи ключа с дополнительной информацией в виде \textbf{одноразовых случайных меток} (nonce -- number used once) вместо меток времени. Протокол передачи состоит в следующем.

\begin{enumerate}
    \item Сторона $A$ вырабатывает случайное число $r_A$, шифрует сообщение, в котором  $(r_A, A)$ -- реквизиты $A$, и передает его стороне $B$:
            \[ A \rightarrow B: ~ E_{K_{AB}}(r_A, A). \]
    \item Сторона $B$ вырабатывает сеансовый ключ $K$, создает шифрованное сообщение и посылает его $A$:
            \[ A \leftarrow B: ~ E_{K_{AB}}(K, r_A, A). \]
    \item Сторона $A$ расшифровывает полученное сообщение
            \[ D_{K_{AB}}( E_{K_{AB}}( K, r_A, A)) = (K, r_A, A). \]
        В результате $A$ получает сеансовый ключ и подтверждение своих реквизитов, что является дополнительной аутентификацией.
\end{enumerate}

Предположим, что сторона $B$ тоже желает убедиться, что имеет дело со стороной $A$. Тогда этот протокол следует дополнить передачей реквизитов $B$. По-прежнему считаем, что у $A$ и $B$ -- общая система шифрования с долговременным секретным ключом $K_{AB}$.

\begin{enumerate}
    \item Сторона $A$ вырабатывает случайное число $r_A$, шифрует и передает стороне $B$ сообщение, в котором  $(r_A, A)$ -- реквизиты $A$:
            \[ A \rightarrow B: ~ E_{K_{AB}}(r_A, A). \]
    \item Сторона $B$ вырабатывает случайное число $r_B$ и отправляет стороне $A$ зашифрованное сообщение:
            \[ A \leftarrow B: ~ E_{K_{AB}}(K_B, r_B, r_A, A), \]
        где $K_B$ -- ключ $B$.
     \item Сторона $A$ осуществляет расшифрование
            \[ D_{K_{AB}}(K_B, r_B, r_A, A) = (K_B, r_B, r_A, A) \]
        и получает ключ $K_B$, и реквизиты $r_B, r_A, A$. Для аутентификации себя сторона $A$ создает свой ключ $K_A$ и отправляет стороне $B$ шифрованное сообщение
            \[ A \rightarrow B: ~ E_{K_{AB}}(K_A, r_B, r_A, B). \]
     \item Сторона $B$ осуществляет расшифрование
            \[ D_{K_{AB}}(K_A, r_B, r_A, B) = (K_A, r_B, r_A, B), \]
        которое определяет ключ $K_A$ и аутентифицирует $A$.
\end{enumerate}

Таким образом, обе стороны имеют в своем распоряжении ключи $K_A, K_B$ в качестве сеансовых секретных ключей.


\subsection{Протокол с ключевым кодом аутентификации}

При использовании хэш-функции $K = h(K_{A} \| K_{B})$ происходит усиление секретности. Здесь $(K_{A} \| K_{B})$ -- конкатенация $K_{A} $ и $K_{B}$.

% Достоинства: предположим, $K_{A} ,K_{B} $ - не обладают «хорошими» свойствами случайности (биты распределены неравномерно или зависимы друг от друга), т.е., $P_{K_{A} ,K_{B} } (0)=\frac{1}{2} -\varepsilon $, где $\varepsilon $ - мало, но не 0. Тогда вероятность того, что этот бит в \textit{K }будет равным нулю, $P_{K} (0)=\frac{1}{2} -\varepsilon ',\varepsilon '<\varepsilon $- усиление секретности.

Вычисление хэш-значения, как правило, выполняется быстрее, чем расшифрование. Поэтому были разработаны протоколы, в которых вместо функции шифрования используется ключевой код аутентификации на основе хэш-функции $\MAC_K$. Рассмотрим протокол такого рода.
\begin{enumerate}
    \item  Сторона $A$ вырабатывает сеансовый ключ $K$, использует одноразовую метку $t_{A}$, создает и пересылает стороне $B$ сообщение:
            \[ A \rightarrow B: ~ t_A, ~ B, ~ K \oplus \MAC_{K_{AB}}( t_A, B), ~ \MAC_{K_{AB}}(K, t_A, B). \]
    \item  Сторона $B$ вычисляет
            \[ \MAC_{K_{AB}}(t_A, B) \oplus K \oplus \MAC_{K_{AB}}(t_A, B) = K \]
        и получает сеансовый ключ $K$.
\end{enumerate}

Заметим, что криптоаналитик может добавить в поле случайную последовательность, тогда вместо $K$ получаем <<$K$ плюс помеха>>. Вмешательство криптоаналитика будет выявлено благодаря наличию четвертого поля в сообщении. Используя полученное значение $K$, вычисляют $\MAC_{K_{AB}}(K, t_A, B)$ и сравнивают с четвертым полем. Если совпадает, то вмешательства криптоаналитика не было.

\input{needham-schroeder_protocol}

\section{Протоколы на криптосистемах с открытым ключом}

\subsection{Простой протокол}

Рассмотрим протокол распространения ключей с помощью асимметричных шифров. Введем обозначения: $K_B$ -- открытый ключ стороны $B$, а $K_A$ -- открытый ключ стороны $A$. Протокол включает три сеанса обмена информацией.
\begin{enumerate}
    \item В первом сеансе сторона $A$ посылает стороне $B$ сообщение
            \[ A \rightarrow B: ~ E_{K_B}(K_1, A), \]
        где $K_1$ -- ключ, выработанный стороной $A$.
    \item Сторона $B$ получает $(K_1, A)$ и передает стороне $A$ наряду с другой информацией свой ключ $K_2$ в сообщении, зашифрованном с помощью открытого ключа $K_A$:
            \[ A \leftarrow B: ~ E_{K_A}(K_2, K_1, B). \]
    \item Сторона $A$ получает и расшифровывает сообщение $(K_2, K_1, B)$. Во время третьего сеанса  сторона $A$, чтобы подтвердить, что она знает ключ $K_2$, посылает стороне $B$ сообщение
            \[ A \rightarrow B: ~ E_{K_B}(K_2). \]
\end{enumerate}
Общий ключ формируется из двух ключей $K_1, K_2$.

\subsection{Протоколы с цифровыми подписями}

Существуют протоколы обмена, в которых перед началом обмена ключами генерируются подписи сторон $A$ и $B$, соответственно $S_A(m)$ и $S_B(m)$. В этих протоколах можно использовать различные одноразовые метки. Рассмотрим пример.
\begin{enumerate}
    \item Сторона $A$ выбирает ключ $K$ и вырабатывает сообщение
            \[ \left( K, ~ t_A, ~ S_A(K, t_A, B) \right), \]
        где $t_A$ -- метка времени. Зашифрованное сообщение передает стороне $B$:
        \[ A \rightarrow B: ~ E_{K_B}(K, ~ t_A, ~ S_A(K, t_A, B)). \]
    \item Сторона $B$ получает $\left( K, ~ t_A, ~ S_A(K, t_A, B) \right)$ и вырабатывает свою метку времени $t_B$. Проверка считается успешной, если $|t_B - t_A | < \delta $. Сторона $B$ знает свои реквизиты и может осуществлять проверку подписи.
\end{enumerate}

Имеется второй вариант протокола, в котором шифрование и подпись выполняются раздельно.
\begin{enumerate}
    \item Сторона $A$ вырабатывает ключ $K$, использует одноразовую метку (или метку времени) $t_{A}$ и передает стороне $B$ два различных зашифрованных сообщения
            \[ \begin{array}{ll}
                A \rightarrow B: & ~ E_{K_B}(K, t_A), \\
                A \rightarrow B: & ~ S_A(K, t_A, B). \\
            \end{array} \]
    \item Сторона $B$ получает это сообщение, расшифровывает $K, t_A$ и, добавив  свои реквизиты $B$, может проверить подпись $S_A(K, t_A, B)$.
\end{enumerate}

В третьем варианте протокола сначала производится шифрование, потом подпись.
\begin{enumerate}
    \item Сторона $A$ вырабатывает ключ $K$, использует одноразовую случайную метку или метку времени $t_A$ и передает стороне $B$ сообщение
        \[ A \rightarrow B: ~ t_A, ~ E_{K_B}(K, A), ~ S_A(t_A, ~ K, ~ E_{K_B}(K, A)). \]
    \item Сторона $B$ получает это сообщение, расшифровывает $\left( t_A, ~ K, ~ A, ~ E_{K_B}(K, A) \right)$ и проверяет подпись $S_A(t_A, ~ K, ~ E_{K_B}(K, A))$.
\end{enumerate}

\input{diffie-hellman}

%\section{Протоколы с аутентификацией}

\subsection{Односторонняя аутентификация}

\input{el-gamal_protocol}

\input{mti}

\input{sts}

\input{girrault}

В этом разделе были рассмотрены протоколы, в которых ключи вырабатываются в процессе обмена информацией.
%Существует и другой подход, который будет рассмотрен в следующих разделах.

\chapter{Разделение секрета}

\section{Пороговые схемы}

Идея \textbf{пороговой} $(n, N)$-схемы\index{разделение секрета!пороговое} разделения общего секрета среди $N$ пользователей состоит в следующем.
%описывается так:
Доверенная сторона хочет распределить некий секрет $K_0$ между $N$ пользователями таким образом, что:
%. Поставлены следующие условия:
\begin{itemize}
    \item любые $m, ~ n \le m \le N$,  легальных пользователей могут получить секрет (или доступ к секрету), если предъявят свои секретные ключи;
    \item любые $m, ~ m < n$, легальных пользователей не могут получить секрет, и не могут определить (вычислить) этот секрет, пытаясь решить трудную в вычислительном смысле  задачу.
\end{itemize}

Далее рассмотрены два случая: $(n, N)$-схема Шамира и простая $(N,N)$-схема.

\input{shamirs_secret_sharing}

\input{xor_secret_sharing}

\section[Распределение секрета по коалициям]{Распределение секрета по \protect\\ коалициям}

\subsection[Схема для нескольких коалиций]{Распределение секрета по нескольким \protect\\ коалициям}

Предположим, что имеется $N$ легальных пользователей
    \[ \{ U_1, U_2, \dots, U_N \}, \]
которым нужно сообщить (открыть, получить доступ к) общий секрет $K$.

Секрет может быть доступен только определенным  коалициям\index{распределение секрета!по коалициям}, например
\[ \begin{array}{l}
    C_1 = \{ U_1, U_2 \}, \\
    C_2 = \{ U_1, U_3, U_4 \}, \\
    C_3 = \{ U_2, U_3 \}, \\
    \dots
\end{array} \]
При этом ни одна из коалиций $C_i, ~ i = 1, 2, \dots$ не должна быть подмножеством другой коалиции.


\subsubsection{Пример 1}

Имеется 4 участника
    \[ \{ U_1, U_2, U_3, U_4 \}, \]
которые образуют 3 коалиции
\[ \begin{array}{l}
    C_1 = \{ U_1, U_2 \}, \\
    C_2 = \{ U_1, U_3 \}, \\
    C_3 = \{ U_2, U_3, U_4 \}. \\
\end{array} \]
Распределение частичных секретов между ними представлено в виде табл. \ref{tab:secret-share-coalition-1}, в которой введены следующие обозначения: $a_1, b_1, c_2, c_3$ -- случайные числа из кольца $\Z_M$. В строках таблицы содержатся частичные секреты каждого из пользователей, в столбцах таблицы показаны частичные секреты, соответствующие каждой из коалиций.

\begin{table}[h!]
    \centering
    \caption{Распределение секрета по определенным коалициям\label{tab:secret-share-coalition-1}}
    \begin{tabular}{|c||c|c|c|}
        \hline
              & $C_1 = \{ U_1, U_2 \}$ & $C_2 = \{U_1, U_3 \}$ & $C_3 = \{ U_2, U_3, U_4 \}$ \\
        \hline \hline
        $U_1$ & $a_1$     & $b_1$     & -- \\
        $U_2$ & $K - a_1$ & --        & $c_2$ \\
        $U_3$ & --        & $K - b_1$ & $c_3$  \\
        $U_4$ & --        & --        & $K - c_2 - c_3$ \\
        \hline
    \end{tabular}
\end{table}

Как видно из приведенных данных, суммирование по модулю $M$ чисел, приведенных в каждом из столбцов таблицы, открывает секрет $K$.


\subsubsection{Пример 2}

%\section{Схема разделения секрета на монотонных булевых функциях}
%\example
В системе распределения секрета доверенный
%с использованием монотонных булевых функций
центр использует кольцо $\Z_m$ целых чисел по модулю $m$. Требуется разделить секрет $K$ между $5$ пользователями
    \[ \{ U_1, U_2, U_3, U_4, U_5 \} \]
так, чтобы восстановить секрет могли только коалиции
\[ \begin{array}{lll}
    C_1 = \{ U_1, U_2 \},      & & C_2 = \{ U_1, U_3 \}, \\
    C_3 = \{ U_2, U_3, U_4 \}, & & C_4 = \{ U_2, U_3, U_5 \}, \\
    C_5 = \{ U_3, U_4, U_5 \}, & & C_6 = \{ U_1, U_2, U_3 \}. \\
\end{array} \]

Заданное множество коалиций с доступом не является минимальным, так как одни коалиции входят в другие:
    \[ C_1 \subset C_6, ~ C_2 \subset C_6. \]
Исключая $C_6$, получим минимальное множество коалиций с доступом к секрету -- ни одна из оставшихся коалиций не входит в другую $C_i \nsubseteq C_j$ для $i \neq j$. Пользователям выдаются тени по минимальному множеству коалиций с доступом. В строках таблицы \ref{tab:secret-share-coalition-2} содержатся частичные секреты каждого из пользователей, в столбцах таблицы показаны частичные секреты, соответствующие каждой из коалиций.

\begin{table}[h!]
    \centering
    \caption{Распределение секрета по определенным коалициям\label{tab:secret-share-coalition-2}}
    \begin{tabular}{|c||c|c|c|c|c|}
        \hline
              & $C_1$     & $C_2$     & $C_3$           & $C_4$           & $C_5$  \\
        \hline \hline
        $U_1$ & $a_1$     & $b_1$     & --              & --              & -- \\
        $U_2$ & $K - a_1$ & --        & $c_2$           & $d_2$           & --\\
        $U_3$ & --        & $K - b_1$ & $c_3$           & $d_3$           & $e_3$ \\
        $U_4$ & --        & --        & $K - c_2 - c_3$ & --              & $e_4$ \\
        $U_5$ & --        & --        & --              & $K - d_2 - d_3$ & $K - e_3 - e_4$ \\
        \hline
    \end{tabular}
\end{table}

Тени выбираются случайно из кольца $\mathbb{\Z}_m$. В результате у пользователей будут тени:
%\exampleend

\input{brickells_scheme}

\subsection{Схема Блома распределения парных ключей}
\selectlanguage{russian}

Рассмотрим распределение ключей по \textbf{схеме Блома}\index{распределение секрета!Блома} (Blom), в которой каждые два пользователя из общего числа $N$ пользователей имеют доступ к общему секретному ключу, ключи различных пар различны.

По-прежнему
    \[ \{ U_1, U_2, \dots, U_N \} \]
-- легальные пользователи, $\Z_p$ -- кольцо целых чисел.

Построим симметричный многочлен
    \[ f(x,y) = \sum_{i=1}^k \sum_{j=1}^k a_{ij} x^i y^j, \]
    \[ a_{ij} \in \Z_p, ~ a_{ij} = a_{ji}. \]

Возьмем набор чисел $r_1, r_2, \dots, r_N$, где $r_i$ -- открытый ключ пользователя $U_i$, ~ $r_i \in \Z_p$.

Каждый пользователь $U_i$ получает многочлен $f(x,y)$ и вместо $y$ подставляет свое значение $r_i$, так что получается $N$ многочленов $f(x, r_i), ~ i = 1, 2, \dots, N$.

Каждые два участника коалиции должны иметь общий ключ. Пусть, например, $U_1$ и $U_2$ хотят создать общий ключ. Тогда пользователь $U_1$, используя $f(x, r_1)$ и зная $r_2$, вычисляет
    \[ K_{12} = f(r_2, r_1). \]

Пользователь $U_2$, используя $f(x, r_2)$ и зная $r_1$, вычисляет
    \[ K_{1,2} = f(r_1, r_2). \]

Так как для выбранного многочлена справедливо равенство
    \[ f(r_1, r_2) = f(r_2, r_1), \]
то
    \[ K_{12} = K_{21}. \]
Таким образом, два участника коалиции создали общий ключ. Таким же образом поступают и другие пары пользователей. Третий пользователь, не участник коалиции, не может подобрать ключ, так как это представляет собой трудную задачу в вычислительном смысле.

%\section{Пример предварительного распределения ключей и разделения секрета в схеме Блома в виде многочленов}
\example
В схеме распределения ключей Блома для $N=4$ пользователей доверенный Центр выбирает:
\begin{enumerate}
    \item модуль $p = 17$ поля $\GF{p}$;
    \item свой секретный симметричный многочлен от двух переменных
        \[ f(x,y) = a + b (x + y) + c x y \mod p \]
        над полем $\GF{p}$;
    \item открытые ключи для каждого пользователя
        \[ r_1 = 5, ~ r_2 = 9, ~ r_3 = 14, ~ r_4 = 3; \]
    \item вычисляет и секретно раздает многочлен $S_i(x)$ каждому пользователю $U_i$:
        \[ \begin{array}{l}
            S_1(x) = f(x, r_1) = 1 + 2x \mod p, \\
            S_2(x) = f(x, r_2) = 3 + 10x \mod p, \\
            S_3(x) = f(x, r_3) = 14 + 3x \mod p, \\
            S_4(x) = f(x, r_4) = 0 + 15x \mod p. \\
        \end{array} \]
\end{enumerate}
Найдем ключи и восстановим секретный многочлен доверенного Центра.

Секретные сеансовые ключи пользователей равны
    \[ K_{ij} = K_{ji} = S_i(r_j) = S_j(r_i): \]
\[ \begin{array}{lcl}
    K_{1,2} = K_{2,1} = 2, & & K_{1,3} = K_{3,1} = 12, \\
    K_{1,4} = K_{4,1} = 7, & & K_{2,3} = K_{3,2} = 7, \\
    K_{2,4} = K_{4,2} = 16, & & K_{3,4} = K_{4,3} = 6. \\
\end{array} \]

По любым 3 многочленам пользователей можно восстановить секретный многочлен Центра. Коэффициенты секретного многочлена Центра равны $a=7, b=9, c=2$.
\exampleend

%\section{Схема предварительного распределения ключей и разделения секрета Блома}
%
%Центр выбирает секретную симметрическую $(k \times k)$-матрицу $F$ над полем $\GF{p}$, где $p$ -- простое. Каждому пользователю $i$ Центр создает и выдает открытый ключ $P_i$, который является $k$-мерным вектором над $\GF{p}$, и секретный ключ $S_i = F \cdot P_i$, $k$-мерный вектор.
%
%Когда два пользователя $i$ и $j$ хотят создать секретный сессионный ключ для обмена сообщениями, они обмениваются открытыми ключами $P_i, P_j$ и вычисляют секретный сессионный ключ $K_{ij} = S_i P_j^T = S_j P_i^T$.
%
%%Если известны открытые ключи $k$ пользователей, то...
%%TODO
%
%%Схема Блома используется в High-bandwidth Digital Content Protection (HDCP), разрботанной Intel для применения в DVD плеерах и телевидении высокой четкости.


\chapter{Примеры систем защиты}

\section{Система Kerberos для локальной сети}
\selectlanguage{russian}

Система аутентификации и распределения ключей Kerberos основана на протоколе Нидхэма--Шредера. Самые известные реализации протокола Kerberos включают Microsoft Active Directory и ПО Kerberos с открытым кодом для Unix.

Протокол предназначен для решения задачи аутентификации и распределения ключей в рамках локальной сети, в которой есть группа пользователей, имеющих доступ к набору сервисов, и требуется обеспечить единую аутентификацию для всех сервисов. Протокол Kerberos сделан полностью на симметричных криптосистемах. Секретный ключ используется для взаимной аутентификации.

Естественно, что в нелокальной сети интернет невозможно секретно создать и распределить пары секретных ключей и поэтому Kerberos построен для (виртуальной) локальной сети.

В протоколе используется 4 типа субъектов:

\begin{itemize}
    \item пользователи системы $C_i$,
    \item сервисы $S_i$, доступ к которым имеют пользователи,
    \item сервер аутентификации AS (authentication server), который производит аутентификацию пользователей по паролям и/или смарт-картам только один раз и выдает секретные сеансовые ключи для дальнейшей аутентификации,
    \item сервер выдачи мандатов TGS (ticket granting server) для аутентификации доступа к запрашиваемым сервисам; аутентификация выполняется по сеансовым ключам\index{ключ!сеансовый}, выданным сервером AS.
\end{itemize}

Для работы протокола требуется заранее распределить следующие секретные симметричные ключи для взаимной аутентификации.
\begin{itemize}
    \item Ключи $K_{C_i}$ между пользователем $i$ и сервером AS. Как правило, ключом служит обычный пароль\index{пароль}, точнее результат хэширования пароля. Может быть использована и смарт-карта.
    \item Ключ $K_{TGS}$ между серверами AS и TGS.
    \item Ключи $K_{S_i}$ между сервисами $S_i$ и сервером TGS.
\end{itemize}

\begin{figure}[h!]
	\centering
	\includegraphics[width=\textwidth]{pic/kerberos}
	\caption{Схема аутентификации и распределения ключей Kerberos\label{fig:kerberos}}
\end{figure}

На рис. \ref{fig:kerberos} представлена схема протокола, состоящая из 6 шагов.

Введем обозначения для протокола между пользователем $C$ с ключом $K_C$ и сервисом $S$ с ключом $K_S$.
\begin{itemize}
    \item $ID_C, ID_{TGS}, ID_S$ -- идентификаторы пользователя, сервера TGS и сервиса $S$, соответственно,
    \item $t_i, \tilde{t}_i$ -- запрашиваемые и выданные границы времени действия сеансовых ключей аутентификации,
    \item $ts_i$ -- метка текущего времени (timestamp),
    \item $N_i$ -- одноразовая метка (nonce)\index{одноразовая метка} - псевдослучайное число, для защиты от атак воспроизведения сообщений,
    \item $K_{C,TGS}, K_{C,S}$ -- выданные сеансовые ключи аутентификации пользователя и сервера TGS, пользователя и сервиса $S$, соответственно,
    \item $T_{TGS} = E_{K_{TGS}}(K_{C,TGS} ~\|~ ID_C ~\|~ \tilde{t}_1)$ -- мандат (ticket) для TGS, который пользователь расшифровать не может.
    \item $T_{S} = E_{K_S}(K_{C,S} ~\|~ ID_C ~\|~ \tilde{t}_2)$ -- мандат для сервиса $S$, который пользователь расшифровать не может.
    \item $K_1, K_2$ -- обмен информацией для генерирования общего секретного симметричного ключа дальнейшей коммуникации, например, по протоколу Диффи--Хеллмана.
\end{itemize}

Схема протокола следующая.
\begin{enumerate}
    \item Первичная аутентификация пользователя по паролю, получение сеансового ключа $K_{C,TGS}$ для дальнейшей аутентификации. Это действие выполняется один раз для каждого пользователя, чтобы уменьшить риск компроментации пароля.
        \begin{enumerate}
            \item $C \rightarrow AS: ~~ ID_C ~\|~ ID_{TGS} ~\|~ t_1 ~\|~ N_1$.
            \item $C \leftarrow AS: ~~ ID_C ~\|~ T_{TGS} ~\|~ E_{K_C}( K_{C,TGS} ~\|~ \tilde{t}_1 ~\|~ N_1 ~\|~ ID_{TGS})$.
        \end{enumerate}
    \item Аутентификация сеансовым ключом $K_{C,TGS}$ на сервере TGS для запроса доступа к сервису выполняется один раз для каждого сервиса. Получение другого сеансового ключа аутентификации $K_{C,S}$.
        \begin{enumerate}
            \item $C \rightarrow TGS: ~~ ID_S ~\|~ t_2 ~\|~ N_2 ~\|~ T_{TGS} ~\|~ E_{K_{C,TGS}}(ID_C ~\|~ ts_1)$.
            \item $C \leftarrow TGS: ~~ ID_C ~\|~ T_{S} ~\|~ E_{K_{C,TGS}}( K_{C,S} ~\|~ \tilde{t}_2 ~\|~ N_2 ~\|~ ID_S)$.
        \end{enumerate}
    \item Аутентификация сеансовым ключом $K_{C,S}$ на сервисе $S$ -- создание общего сеансового ключа дальнейшего взаимодействия.
        \begin{enumerate}
            \item $C \rightarrow S: ~~ T_{S} ~\|~ E_{K_{C,S}}(ID_C ~\|~ ts_2 ~\|~ K_1)$.
            \item $C \leftarrow S: ~~ E_{K_{C,S}}( ts_2 ~\|~ K_2)$.
        \end{enumerate}
\end{enumerate}

Аутентификация и проверка целостности достигается сравнением идентификаторов, одноразовых меток и меток времени внутри зашифрованных сообщений после расшифрования с их действительными значениями.

Некоторым недостатком схемы является необходимость синхронизации часов между субъектами сети.


\section{Инфраструктура открытых ключей}

\subsection{Иерархия удостоверяющих центров}
\selectlanguage{russian}

Проблему аутентификации и распределения сеансовых симметричных ключей шифрования в Интернете, а также в больших локальных и виртуальных сетях решают с помощью построения иерархии открытых ключей и криптосистем с открытым ключом.

\begin{enumerate}
    \item Существует удостоверяющий центр (УЦ) верхнего уровня, корневой УЦ, (Root Certification Authority, $CA$), обладающий парой из секретного и открытого ключей. Открытый ключ УЦ верхнего уровня распространяется среди всех пользователей, причем все пользователи \emph{доверяют УЦ}. Это означает, что
        \begin{itemize}
            \item УЦ -- <<хороший>>, обеспечивает надежное хранение секретного ключа, не пытается фальсифицировать и скомпрометировать свои ключи,
            \item имеющийся у пользователей открытый ключ УЦ действительно принадлежит УЦ.
        \end{itemize}
        В массовых информационных и интернет-систем открытые ключи многих корневых УЦ встроены в дистрибутивы и пакеты обновлений ПО. Доверие пользователей неявно проявляется в их уверенности в том, что открытые ключи корневых УЦ, включенные в ПО, не фальсифицированы и не скомпрометированы. \emph{Де-факто пользователи доверяют а) распространителям ПО и обновлений, б) корневому УЦ.}\index{доверие}

        Назначение УЦ верхнего уровня -- проверка принадлежности и подписание открытых ключей других удостоверяющих центров второго уровня, а также организаций и сервисов. УЦ подписывает своим секретным ключом следующее сообщение:
        \begin{itemize}
            \item название и URI УЦ нижележащего уровня или организации/сервиса,
            \item значение сгенерированного открытого ключа и название алгоритма соответствующей криптосистемы с открытым ключом,
            \item время выдачи и срок действия открытого ключа.
        \end{itemize}

    \item УЦ второго уровня (certificate authority, CA) имеют свои пары открытых и секретных ключей, сгенерированных и подписанных корневым УЦ. Причем перед подписанием корневой УЦ убеждается в <<надежности>> УЦ второго уровня, производит юридические проверки. Корневой УЦ не имеет доступа к секретным ключам УЦ второго уровня.

        Пользователи, имея в своей базе открытых ключей доверенные открытые ключи корневого УЦ, могут проверить ЭЦП открытых ключей УЦ 2-го уровня и убедиться, что предъявленный открытый ключ действительно принадлежит данному УЦ. Таким образом:
        \begin{itemize}
            \item Пользователи полностью доверяют корневому УЦ и его открытому ключу, который у них хранится. Пользователи верят, что корневой УЦ не подписывает небезопасные ключи и гарантирует, что подписанные им ключи действительно принадлежат УЦ 2-го уровня.
            \item Проверив ЭЦП открытого ключа УЦ 2-го уровня с помощью доверенного открытого ключа УЦ 1-го уровня, пользователь верит, что открытый ключ УЦ  2-го уровня действительно принадлежит данному УЦ и не был скомпрометирован.
        \end{itemize}

        Аутентификация в протоколе защищенного интернет соединения SSL/TLS достигается в результате  проверки пользователями совпадения URI-адреса сервера из ЭЦП с фактическим адресом.

        УЦ второго уровня в свою очередь тоже подписывает открытые ключи УЦ третьего уровня, а также организаций.  И так далее по уровням.

    \item В результате построена \emph{иерархия} подписанных открытых ключей.

    \item Открытый ключ с идентификационной информацией (название организации, URI-адрес вебресурса, дата выдачи, срок действия и др.) и подписью УЦ вышележащего уровня, заверяющей ключ и идентифицирующие реквизиты, называется \textbf{сертификатом открытого ключа},\index{сертификат открытого ключа} на который существует международный стандарт X.509, последняя версия 3. В сертификате указывается его область применения: подписание других сертификатов, аутентификация для веба, аутентификация для электронной почты и т.д.
\end{enumerate}


\begin{figure}[h!]
	\centering
	\includegraphics[width=0.8\textwidth]{pic/X509-hierarchy}
	\caption{Иерархия сертификатов\label{fig:x509-hierarchy}}
\end{figure}

На рис. \ref{fig:x509-hierarchy} приведены пример иерархии сертификатов и путь подписания сертификата X.509 интернет-сервиса Google Mail.

Система распределения, хранения и управления сертификатами открытых ключей называется \textbf{инфраструктурой открытых ключей}\index{инфраструктура открытых ключей} (public key infrastructure, PKI)\index{PKI}. PKI применяется для аутентификации в системах SSL, IPsec, PGP и т.д. Помимо процедур выдачи и распределения открытых ключей PKI также определяет процедуру отзыва скомпрометированных или устаревших сертификатов.


\subsection{Структура сертификата X.509}
\selectlanguage{russian}

Ниже приведен пример сертификата X.509\index{X509.3} интернет-сервиса mail.google.com, используемый для защищенного SSL-соединения в 2009 г. Сертификат напечатан командой \texttt{openssl x509 -in file.crt -noout -text}:

{\small \begin{verbatim}
Certificate:
Data:
  Version: 3 (0x2)
  Serial Number:
    6e:df:0d:94:99:fd:45:33:dd:12:97:fc:42:a9:3b:e1
  Signature Algorithm: sha1WithRSAEncryption
  Issuer: C=ZA, O=Thawte Consulting (Pty) Ltd.,
    CN=Thawte SGC CA
  Validity
    Not Before: Mar 25 16:49:29 2009 GMT
    Not After : Mar 25 16:49:29 2010 GMT
  Subject: C=US, ST=California, L=Mountain View, O=Google Inc,
    CN=mail.google.com
  Subject Public Key Info:
    Public Key Algorithm: rsaEncryption
    RSA Public Key: (1024 bit)
      Modulus (1024 bit):
        00:c5:d6:f8:92:fc:ca:f5:61:4b:06:41:49:e8:0a:
        2c:95:81:a2:18:ef:41:ec:35:bd:7a:58:12:5a:e7:
        6f:9e:a5:4d:dc:89:3a:bb:eb:02:9f:6b:73:61:6b:
        f0:ff:d8:68:79:1f:ba:7a:f9:c4:ae:bf:37:06:ba:
        3e:ea:ee:d2:74:35:b4:dd:cf:b1:57:c0:5f:35:1d:
        66:aa:87:fe:e0:de:07:2d:66:d7:73:af:fb:d3:6a:
        b7:8b:ef:09:0e:0c:c8:61:a9:03:ac:90:dd:98:b5:
        1c:9c:41:56:6c:01:7f:0b:ee:c3:bf:f3:91:05:1f:
        fb:a0:f5:cc:68:50:ad:2a:59
      Exponent: 65537 (0x10001)
  X509v3 extensions:
    X509v3 Extended Key Usage: TLS Web Server
      Authentication, TLS Web Client Authentication,
      Netscape Server Gated Crypto
    X509v3 CRL Distribution Points:
    URI:http://crl.thawte.com/ThawteSGCCA.crl
    Authority Information Access:
    OCSP - URI:http://ocsp.thawte.com
    CA Issuers - URI:http://www.thawte.com/repository/
        Thawte_SGC_CA.crt
    X509v3 Basic Constraints: critical
    CA:FALSE
Signature Algorithm: sha1WithRSAEncryption
  62:f1:f3:05:0e:bc:10:5e:49:7c:7a:ed:f8:7e:24:d2:f4:a9:
  86:bb:3b:83:7b:d1:9b:91:eb:ca:d9:8b:06:59:92:f6:bd:2b:
  49:b7:d6:d3:cb:2e:42:7a:99:d6:06:c7:b1:d4:63:52:52:7f:
  ac:39:e6:a8:b6:72:6d:e5:bf:70:21:2a:52:cb:a0:76:34:a5:
  e3:32:01:1b:d1:86:8e:78:eb:5e:3c:93:cf:03:07:22:76:78:
  6f:20:74:94:fe:aa:0e:d9:d5:3b:21:10:a7:65:71:f9:02:09:
  cd:ae:88:43:85:c8:82:58:70:30:ee:15:f3:3d:76:1e:2e:45:
  a6:bc
\end{verbatim}}

Как видно, сертификат действителен с 26.03.2009 до 25.03.2010, открытый ключ представляет собой ключ RSA с длиной модуля $n=$ 1024 бит и экспонентой $e = 65537$ и принадлежит компании Google Inc. Открытый ключ предназначен для взаимной аутентификации веб-сервера mail.google.com и веб-клиента в протоколе SSL/TLS. Сертификат подписан секретным ключом удостоверяющего центра Thawte SGC CA, подпись вычислена с помощью криптографического хэша SHA-1 и алгоритма RSA. В свою очередь сертификат с открытым ключом Thawte SGC CA для проверки значения ЭЦП данного сертификата расположен по адресу http://www.thawte.com/repository/Thawte\_SGC\_CA.crt.

ЭЦП вычисляется от всех полей сертификата, кроме самого значения подписи.


\input{pgp}

\input{tls}

\input{ipsec}

\section[Защита персональных данных в мобильной связи]{Защита персональных данных в \protect\\ мобильной связи}

\input{gsm2}

\input{gsm3}

%\section{Беспроводная сеть Wi-Fi}
%\subsection{WPA-PSK2, 802.11n, Radix?}
%\subsection{Wimax 802.16(?)}

\chapter{Аутентификация пользователя}


\section{Многофакторная аутентификация}

Для защищенных приложений применяется \textbf{многофакторная} аутентификация одновременно по факторам различной природы:
\begin{enumerate}
    \item Свойство, которым обладает субъект. Например, биометрия, природные уникальные отличия: лицо, отпечатки пальцев, радужная оболочка глаз, капиллярные узоры, последовательность ДНК.
    \item Знание -- информация, которую знает субъект. Например, пароль, пин-код.
    \item Владение -- вещь, которой обладает субъект. Например, электронная или магнитная карта, флеш-память.
%    \item Факторы присвоения. Например, номер машины, RFID-метка.
\end{enumerate}

В обычных массовых приложениях, из-за удобства использования, применяется аутентификация только по \textbf{паролю}\index{пароль}, который является общим секретом пользователя и информационной системы. Биометрическая аутентификация по отпечаткам пальцев применяется существенно реже. Как правило, аутентификация по отпечаткам пальцев является дополнительным, а не вторым обязательным фактором (тоже из-за удобства ее использования).

%Так же явно или неявно используется аутентификация по факторам:
%\begin{enumerate}
%    \item Социальная сеть. Доверие к индивидууму в личном общении или интернет на основании общих связей.
%    \item Географическое положение. Например, для проверки оплаты товаров по кредитной карте.
%    \item Время. Доступ к сервисам или местам только в определенное время.
%    \item И др.
%\end{enumerate}


\section[Энтропия и криптостойкость паролей]{Энтропия и криптостойкость \protect\\ паролей}

Стандартный набор символов паролей, которые можно набрать на клавиатуре, используя английские буквы и небуквенные символы, состоит из $D=94$ символов. При длине пароля $L$ символов и предположении равновероятного использования символов энтропия паролей равна
    \[ H = L \log_2 D. \]

К. Шеннон, исследуя энтропию символов английского текста, исследовал вероятность успешного предсказания людьми следующего символа по первым нескольким символам слов или текста. В результате К. Шеннон получил оценку энтропию первого символа $s_1$ текста порядка $H(s_1) \approx 4.6$--$4.7$ бит/символ, и оценки энтропии последующих символов постепенно уменьшающиеся до $H(s_9) \approx 1.5$ бит/символ для 9-го символа. Энтропия для длинных текстов литературных произведений получила оценку $H(s_\infty) \approx 0.4$ бит/символ.

Статистические исследования баз паролей показывают, что наиболее часто используются буквы -- $a,e,o,r$, и  цифра -- $1$.

NIST использует следующие рекомендации для оценки энтропии паролей\index{энтропия!пароля}, создаваемых людьми.
\begin{enumerate}
    \item Энтропия первого символа $H(s_1) = 4$ бит/символ.
    \item Энтропия со 2-го по 8-й символы $H(s_{2 \leq i \leq 8}) = 2$ бит/символ.
    \item Энтропия с 9-го по 20-й символы $H(s_{9 \leq i \leq 20}) =   1.5$ бит/символ.
    \item Энтропия с 21-го символа $H(s_{i \geq 21}) = 1$ бит/символ.
    \item Проверка композиции на использование символов разных регистров и небуквенных символов добавляет до 6 бит энтропии пароля.
    \item Словарная проверка на слова и часто используемые пароли добавляет до 6 бит энтропии для коротких паролей. Для 20-символьных и более длинных паролей прибавка к энтропии   0 бит.
\end{enumerate}

Для оценки энтропии пароля нужно сложить энтропии символов $H(s_i)$ и сделать дополнительные надбавки, если пароль удовлетворяет тестам на композицию и отсутствие в словаре.

\begin{table}[h!]
    \centering
    \caption{Оценка NIST предполагаемой энтропии паролей\label{tab:password-entropy}}
    \resizebox{\textwidth}{!}{ \begin{tabular}{|c||c|c|c||c|}
        \hline
        \multirow{2}{*}{\parbox{1.5cm}{Длина пароля, символы}} & \multicolumn{3}{|c||}{\parbox{6cm}{Энтропия паролей пользователей по критериям NIST}} & \multirow{2}{*}{\parbox{2.5cm}{Энтропия случайных равновероятных паролей}} \\
        \cline{2-4}
        & \parbox{1.5cm}{Без проверок} & \parbox{2cm}{Словарная проверка} & \parbox{2.5cm}{Словарная и композиционная проверка} & \\
        \hline
        4  & 10 & 14 & 16 & 26.3 \\
        6  & 14 & 20 & 23 & 39.5 \\
        8  & 18 & 24 & 30 & 52.7 \\
        10 & 21 & 26 & 32 & 65.9 \\
        12 & 24 & 28 & 34 & 79.0 \\
        16 & 30 & 32 & 38 & 105.4 \\
        20 & 36 & 36 & 42 & 131.7 \\
        24 & 40 & 40 & 46 & 158.0 \\
        30 & 46 & 46 & 52 & 197.2 \\
        40 & 56 & 56 & 62 & 263.4 \\
        \hline
    \end{tabular} }
\end{table}

В табл. \ref{tab:password-entropy} приведена оценка NIST на величину энтропии пользовательских паролей в зависимости от их длины и сравнение с энтропией случайных паролей с равномерным распределением символов из набора в $D=94$ символов клавиатуры. Вероятное число попыток для подбора пароля составляет $O(2^H)$. Из таблицы видно, что по критериям NIST энтропия реальных паролей в 2--4 раза меньше энтропии случайных паролей с равномерным распределением символов.

\example
Оценим общее количество существующих паролей. Население Земли -- 7 млрд. Предположим, что все население использует компьютеры, Интернет и у каждого человека по 10 паролей. Общее количество существующих паролей $7 \cdot 10^{10} \approx 2^{36}$.
%Следовательно, \emph{реальная энтропия паролей не превышает 36 бит}.

Имея доступ к наиболее массовым интернет-сервисам с количеством пользователей десятки и сотни миллионов, в которых пароли часто хранятся в открытом виде из-за необходимости обновления ПО и, в частности, выполнения аутентификации, мы 1) имеем базу паролей, покрывающую существенную часть пользователей, 2) можем статистически построить правила генерирования паролей. Даже если пароль хранится в защищенном виде, то при вводе пароль, как правило, в открытом виде пересылается по Интернету, и все преобразования пароля для аутентификации осуществляет интернет-сервис, а не веб-браузер. Следовательно, интернет-сервис имеет доступ к исходному паролю.
\exampleend

В 2002 г. был подобран ключ для 64-битового блокового шифра RC5 сетью \texttt{distributed.net} персональных компьютеров, выполнявших вычисления в фоновом режиме. Суммарное время вычислений всех компьютеров -- 1757 дней, было проверено 83\% пространства всех ключей. Это означает, что пароли с оценочной энтропией менее 64 бит, то есть \emph{все пароли} до 40 символов по критериям NIST,  могут быть подобраны в настоящее время. Конечно, с оговорками на то, что 1) нет ограничений на количество и скорость попыток аутентификаций, 2) алгоритм генерирования вероятных паролей эффективен.

Строго говоря, использование даже 40-символьного пароля для аутентификации или в качестве ключа блокового шифрования является небезопасным.


\subsubsection{Число паролей}

Приведем различные оценки числа паролей, создаваемых людьми.

Пароли, создаваемые людьми, основаны на словах или закономерностях естественного языка. В английском языке всего около $1\ 000\ 000 \approx 2^{20}$ слов, включая термины.

%http://www.springerlink.com/content/bh216312577r6w64/fulltext.pdf
%http://www.antimoon.com/forum/2004/4797.htm

Используемые слоги английского языка имеют вид V, CV, VC, CVV, VCC, CVC, CCV, CVCC, CVCCC, CCVCC, CCCVCC, где C -- согласная (consonant), V -- гласная (vowel). 70\% слогов имеют структуру VC или CVC. Общее число слогов $S = 8000 - 12000$. Средняя длина слога -- 3 буквы.

Предполагая равновероятное распределение всех слогов английского языка, для числа паролей из $r$ слогов получим верхнюю оценку
    \[ N_1 = S^r = 2^{13 r} \approx 2^{4.3 L_1}. \]
Средняя длина паролей составит
    \[ L_1 \approx 3 r. \]

Теперь предположим, что пароли могут состоять только из 2--3 буквенных слогов вида CV, VC, CVV, VCC, CVC, CCV с равновероятным распределением символов. Подсчитаем число паролей $N_2$, которые могут быть построены из $r$ таких слогов. В английском алфавите $n_v = 10, n_c = 16, n = n_v + n_c = 26$. Верхняя оценка числа $r$-слоговых паролей:
    \[ N_2 = (n_c n_v + n_v n_c + n_c n_v n_v + n_v n_c n_c + n_c n_v n_c + n_c n_c n_v)^r \approx \]
        \[ \approx \left( n_c n_v(3 n_c + n_v) \right)^r, \]
    \[ N_2 \approx \left( \frac{n^3}{2} \right)^r \approx 2^{13 r} \approx 2^{4.3 L_2}. \]
Средняя длина паролей:
    \[ L_2 = \frac{n_c n_v(2 + 2 + 3 n_v + 3 n_c + 3 n_c + 3 n_c)}{n_c n_v (1 + 1 + n_v + n_c + n_c + n_c)} \cdot r \approx 3 r. \]

Как видно, получились одинаковые оценки числа и длины паролей.

Подсчитаем верхние оценки числа паролей из $L$ символов, предполагая равномерное распределение символов из алфавита в $D$ символов: a) $D_1 = 26$ строчных буквы, б) все $D_2 = 94$ печатных символа клавиатуры (латиница и небуквенные символы):
    \[ N_3 = D_1^L \approx 2^{4.7 L}, \]
    \[ N_4 = D_2^L \approx 2^{6.6 L}. \]

\begin{table}[h!]
    \centering
    \caption{Различные верхние оценки числа паролей\label{tab:password-number}}
    \resizebox{\textwidth}{!}{ \begin{tabular}{|c||c|c|c|}
        \hline
        \multirow{2}{*}{\parbox{1.5cm}{Длина пароля}} & \multicolumn{3}{|c|}{Число паролей} \\
        \cline{2-4}
            & \parbox{3cm}{На основе слоговой композиции} &
            \parbox{3cm}{Алфавит $D=26$ символов} &
            \parbox{3cm}{Алфавит $D=94$ символа} \\
        \hline \hline
        6  & $2^{26}$ & $2^{28}$ & $2^{39}$ \\
        9  & $2^{39}$ & $2^{42}$ & $2^{59}$ \\
        12 & $2^{52}$ & $2^{56}$ & $2^{79}$ \\
        15 & $2^{65}$ & $2^{71}$ & $2^{98}$ \\
        \hline
        21 & $2^{91}$ & $2^{99}$ & $2^{137}$ \\
        \hline
        39 & $2^{169}$ & $2^{183}$ & $2^{256}$ \\
        \hline
    \end{tabular} }
\end{table}

Из таблицы \ref{tab:password-number} видно, что при доступном объеме вычислений в $2^{60 \ldots 70}$ операций пароли вплоть до 15 символов, построенные на словах, слогах, изменениях слов, вставках цифр, небольшого изменения регистров и другой простейшей , могут быть найдены перебором на кластере (или ПК) в настоящее время.

Для достижения криптостойкости паролей, сравнимой со 128- или 256-битовым секретным ключом, требуется выбирать пароль из 20 и 40 символов соответственно, что, как правило, не реализуется из-за сложности запоминания и ввода без ошибок.


%Подсчитаем число паролей $N_1$, которые могут могут построены из $r$ ~ 2-3 буквенных слогов: $cv, vc, ccv, cvc, vcc$, где $c$ -- согласная, $v$ -- гласная. В английском алфавите $n_v = 10, n_c = 16, n = n_v + n_c = 26$. Число паролей
%    \[ N_1 = \left( n_v n_c (1 + 1 + n_c + n_c + n_c) \right)^r \approx 3^r n_v^r n_c^{2r}. \]
%Средняя длина паролей
%    \[ L = r \left( \frac{2 + 2 + 3 n_c + 3 n_c + 3 n_c}{1 + 1 + n_c + n_c + n_c} \right) \approx 3r. \]
%
%%Учтем, что $b \leq r$ символов могут быть заглавными: $N_1 \rightarrow N_2 < N_1 \binom{L}{b} \left( \frac{n}{n_v} \right)^b$. Вставим $d$ цифр в случайные места: $N_2 \rightarrow N_3 = N_2 (10 (1 + L))^d \approx N_2 (10 L)^d$.
%%
%%Общее число паролей
%%    \[ N = N_3 = 3^r 10^r 16^{2r} \binom{3r}{b} 2.6^b \left(10 \cdot 3 r \right)^d. \]
%%
%%\begin{table}[h!]
%%    \centering
%%    \small
%%    \begin{tabular}{|c|c|c|c|c||cr|}
%%        \hline
%%        \parbox{1.3cm}{Слогов, $r$} & \parbox{1.8cm}{Заглавных букв, $b$} & \parbox{1.5cm}{Вставок цифр, $d$} & \parbox{2.8cm}{Средняя длина пароля, $L+d$} & \parbox{3cm}{Верхняя оценка числа паролей $N$} & \multicolumn{2}{|c|}{\parbox{3.2cm}{Число всех паролей}} \\
%%        \hline
%%        $2$ & $0$ & $0$ & $6$ & $2^{26}$ & $2^{36}$ & a-z \\
%%        $2$ & $2$ & $0$ & $6$ & $2^{32}$ & $2^{48}$ & A-Z, a-z \\
%%        $2$ & $2$ & $2$ & $8$ & $2^{45}$ & $2^{48}$ & A-Z, a-z, 0-9 \\
%%        \hline
%%        $3$ & $0$ & $0$ & $9$ & $2^{39}$ & $2^{54}$ & a-z \\
%%        $3$ & $3$ & $0$ & $9$ & $2^{49}$ & $2^{54}$ & A-Z, a-z \\
%%        $3$ & $3$ & $2$ & $11$ & $2^{63}$ & $2^{65}$ & A-Z, a-z, 0-9 \\
%%        \hline
%%        $4$ & $0$ & $0$ & $12$ & $2^{52}$ & $2^{93}$ & a-z \\
%%        $4$ & $3$ & $0$ & $12$ & $2^{64}$ & $2^{186}$ & A-Z, a-z \\
%%        $4$ & $3$ & $2$ & $14$ & $2^{78}$ & $2^{222}$ & A-Z, a-z, 0-9 \\
%%        \hline
%%    \end{tabular}
%%    \caption{Сравнение верхней оценки числа паролей, построенных на слогах, со всем доступным множеством паролей.}
%%    \label{tab:password-number}
%%\end{table}
%
%Учтем, что $b$ символов в пароле могут быть взяты не из 26-символьного алфавита строчных букв, а из всего алфавита в $D=94$ печатных символа клавиатуры (латиница и небуквенные символы):
%\[
%    \begin{array}{ll}
%    b=1 & N_1 \rightarrow N_2 = \frac{n_v}{n_v+n_c} 3^r n_v^{r-1} n_c^{2r} \cdot L. \]
%
%    \[ N_1 \rightarrow N_2 < N_1 \binom{L}{b} \left( \frac{D}{n_v} \right)^b. \]
%
%
%
%Общее число паролей
%    \[ N < 3^r n_v^r n_c^{2r} \binom{L}{b} \left( \frac{D}{n_v} \right)^b = 3^r 10^r 16^{2r} \binom{3r}{b} \left( \frac{94}{10} \right)^b. \]
%
%\begin{table}[h!]
%    \centering
%    \small
%    \begin{tabular}{|c|c|c|c||cr|}
%        \hline
%        \parbox{1.5cm}{Слогов, $r$} & \parbox{3cm}{Средняя длина пароля, $L$} & \parbox{3cm}{Символов из всего алфавита, $b$} & \parbox{3cm}{Верхняя оценка числа паролей $N$} & \multicolumn{2}{|c|}{\parbox{3.2cm}{Число всех паролей, $D^L$}} \\
%        \hline
%        \multirow{3}{*}{2} & \multirow{3}{*}{6} & $0$ & $2^{26}$ & $2^{28}$ & a-z \\
%        & & $1$ & $2^{32}$ & $2^{34}$ & A-Z, a-z \\
%        & & $3$ & $2^{40}$ & $2^{39}$ & Весь алфавит \\
%        \hline
%        \multirow{3}{*}{3} & \multirow{3}{*}{9} & $0$ & $2^{39}$ & $2^{42}$ & a-z \\
%        & & $2$ & $2^{50}$ & $2^{51}$ & A-Z, a-z \\
%        & & $4$ & $2^{59}$ & $2^{59}$ & Весь алфавит \\
%        \hline
%        \multirow{3}{*}{4} & \multirow{3}{*}{12} & $0$ & $2^{52}$ & $2^{56}$ & a-z \\
%        & & $3$ & $2^{69}$ & $2^{68}$ & A-Z, a-z \\
%        & & $6$ & $2^{81}$ & $2^{77}$ & Весь алфавит \\
%        \hline
%    \end{tabular}
%    \caption{Сравнение верхней оценки числа паролей, построенных на слогах, со всем доступным множеством паролей в алфавите из $D$ символов.}
%    \label{tab:password-number}
%\end{table}
%
%Из таблицы \ref{tab:password-number} видно, что при доступном объеме вычислений в $2^{60 \ldots 70}$ операций, пароли вплоть до 12 символов, построенные на словах, слогах, изменениях слов, вставках цифр, небольшого изменения регистров и другой простейшей обфускации, могут быть найдены перебором на кластере (или ПК) в настоящее время.


\subsubsection{Атака для подбора паролей и ключей шифрования}

В схемах аутентификации по паролю иногда используется хэширование и хранение хэша пароля на сервере. В таких случаях применима словарная атака или атака с применением заранее вычисленных таблиц для ускорения поиска.

Для нахождения пароля, прообраза хэш-функции, или для нахождения ключа блокового шифрования по атаке с выбранным шифротекстом (для одного и того же известного открытого текста и соответствующего шифротекста) может быть применен метод перебора с балансом между памятью и временем вычислений. Самый быстрый метод радужных таблиц (rainbow tables)\index{радужные таблицы}, 2003 г., заранее вычисляет следующие цепочки и хранит результат в памяти.

Для нахождения пароля, прообраза хэш-функции $H$, цепочка строится как
    \[ M_0 \xrightarrow{H(M_0)} h_0 \xrightarrow{R_0(h_0)} M_1 \ldots M_t \xrightarrow{H(M_t)} h_t \xrightarrow{R_t(h_t)} M_{t+1}, \]
где $R_i(h)$ -- функция редуцирования, преобразования хэша в пароль для следующего хэширования.

Для нахождения ключа блокового шифрования для одного и того же известного открытого текста $M$ таблица строится как
    \[ K_0 \xrightarrow{E_{K_0}(M)} c_0 \xrightarrow{R_0(c_0)} K_1 \ldots K_t \xrightarrow{E_{K_t}(M)} c_t \xrightarrow{R_t(c_t)} K_{t+1}, \]
где $R_i(c)$ -- функция редуцирования, преобразования шифротекста в новый ключ.

Функция редуцирования $R_i$ зависит от номера итерации, чтобы избежать дублирующиеся подцепочки, которые возникают в случае коллизий между значениями в разных цепочках в разных итерациях, если $R$ постоянна. Rainbow-таблица размера $(m \times 2)$ состоит из строк $(M_{0,j}, M_{t+1,j})$ или $(K_{0,j}, K_{t+1,j})$, вычисленных для разных значений стартовых паролей $M_{0,j}$ или $K_{0,j}$ соответственно.

Опишем атаку на примере нахождения прообраза $\overline{M}$ хэша $\overline{h} = H(\overline{M})$. На первой итерации исходный хэш $\overline{h}$ редуцируется в сообщение $\overline{h} \xrightarrow{R_t(\overline{h})} \overline{M}_{t+1} $ и сравнивается со всеми значениями последнего столбца $M_{t+1,j}$ таблицы. Если нет совпадения, переходим ко второй итерации. Хэш $\overline{h}$ дважды редуцируется в сообщение $\overline{h} \xrightarrow{R_{t-1}(\overline{h})} \overline{M}_t \xrightarrow{H(\overline{M}_t)} \overline{h}_t \xrightarrow{R_t(\overline{h}_t)} \overline{M}_{t+1}$ и сравнивается со всеми значениями последнего столбца $M_{t+1,j}$ таблицы. Если не совпало, то переходим к третьей итерации и т.д. Если для $r$-кратного редуцирования сообщение $\overline{M}_{t+1}$ содержится в таблице во втором столбце, то из совпавшей строки берется $M_{0,j}$, и вся цепочка пробегается  заново для поиска искомого сообщения $\overline{M}: ~ \overline{h} = H(\overline{M})$.

Найдем вероятность нахождения пароля в таблице. Пусть мощность множества всех паролей $N$. Изначально в столбце $M_{0,j}$ содержится $m_0 = m$ различных паролей. Предполагая случайное отображение с пересечениями паролей $M_{0,j} \rightarrow M_{1,j}$, в $M_{1,j}$ будет $m_1$ различных паролей. Согласно задаче о размещении,
\[
    m_{i+1} = N \left( 1 - \left( 1 - \frac{1}{N} \right)^{m_i} \right) \approx N \left( 1 - e^{-\frac{m_i}{N}} \right).
\]
Вероятность нахождения пароля
\[
    P = 1 - \prod \limits_{i=1}^t \left( 1 - \frac{m_i}{N} \right).
\]

Чем больше таблица из $m$ строк, тем больше шансов найти пароль или ключ, выполнив, в наихудшем случае,   $O \left( m \frac{t(t+1)}{2} \right)$ операций.

Примеры применения атаки на хэш-функциях $\textrm{MD5}$\index{MD5}, $\textrm{LM} \sim \textrm{DES}_{\textrm{Password}} (\textrm{const})$ приведены в табл. \ref{tab:rainbow-tables}.

\begin{table}[h!]
    \centering
    \caption{Атаки на радужных таблицах на \emph{одном} ПК\label{tab:rainbow-tables}}
    \resizebox{\textwidth}{!}{ \begin{tabular}{|c|c|c|c|c|c|c|}
        \hline
        \multirow{2}{*}{\parbox{1.0cm}{Длина, биты}} & \multicolumn{3}{|c|}{Пароль или ключ} &
            \multicolumn{3}{|c|}{Радужная таблица} \\
        \cline{2-7}
        & \parbox{1.2cm}{Длина, симв.} & \parbox{1cm}{Множе- ство} & \parbox{1cm}{Мощн- ость} &
            Объем & \parbox{1.5cm}{Время вычисления таблиц} & \parbox{1.3cm}{Время поиска} \\
        \hline \hline
        \multicolumn{7}{|c|}{Хэш LM} \\
        \hline
        \multirow{3}{*}{$2 \times 56$} & \multirow{3}{*}{14} &
            A--Z & $2^{33}$ & 610 MB &  & 6 с \\
        & & A--Z, 0-9 & $2^{36}$ & 3 GB &  & 15 с \\
        & & все & $2^{43}$ & 64 GB & \parbox{1.5cm}{несколько лет} & 7 мин \\
        \hline \hline
        \multicolumn{7}{|c|}{Хэш MD5} \\
        \hline
        128 & 8 & a-z, 0-9 & $2^{41}$ & 36 GiB & - & 4 мин \\
        \hline
    \end{tabular} }
\end{table}


\section{Аутентификация по паролю}

Из-за малой энтропии пользовательских паролей во всех системах регистрации и аутентификации пользователей применяется специальная политика безопасности. Типичные минимальные требования:
\begin{enumerate}
    \item Длина пароля от 8 символов. Использование разных регистров и небуквенных символов в паролях. Запрет паролей из словаря слов или часто используемых паролей. Запрет паролей в виде дат, номеров машин и других номеров.
    \item Ограниченное время действия пароля. Обязательная смена пароля по истечению срока действия.
    \item Блокирование возможности аутентификации после нескольких неудачных попыток. Ограниченное число актов аутентификаций в единицу времени. Временная задержка перед выдачей результата аутентификации.
\end{enumerate}

Дополнительные рекомендации (требования) пользователям:
\begin{enumerate}
    \item Не использовать одинаковые или похожие пароли для разных систем. Например, электронная почта, вход в ОС, электронная платежная система, форумы, социальные сети. Пароль часто передается в открытом виде по сети. Пароль доступен администратору системы, возможны утечки конфиденциальной информации с серверов. Стараться выбирать случайные стойкие пароли.
    \item Не записывать пароли. Никому не сообщать пароль, даже администратору. Не передавать пароли по почте, телефону, Интернету и т.д.
    \item Не использовать одну и ту же учетную запись для разных пользователей, даже в виде исключения.
    \item Всегда блокировать компьютер, когда пользователь отлучается от него даже на короткое время.
\end{enumerate}


\section[Хранение паролей и аутентификация в ОС]{Хранение паролей и \protect\\ аутентификация в ОС}

Для усложнения подбора пароля и избежания словарной атаки используется добавление перед хэшированием к паролю соли, случайной битовой строки. \textbf{Солью} (salt)\index{соль} называется (псевдо) случайная битовая строка $s$, добавляемая к аргументу $m$ (паролю) функции хэширования $h(m)$ для рандомизации хэширования одинаковых сообщений. Соль передается (хранится) вместе с вычисленным хэшем для возможности повторных вычислений:
    \[ s ~\|~ h(s ~\|~ m). \]

Соль применяется для избежания словарных атак.

\textbf{Словарная} атака заключается в том, что злоумышленник один раз заранее вычисляет таблицы хэшей от наиболее \emph{вероятных} сообщений, т.е. составляет словарь, и далее производит поиск по вычисленной таблице для взламывания исходного сообщения. Словарные атаки использовались ранее для взлома паролей $m$, которые хранились в виде обычных хэшей $h(m)$. Усовершенствованной словарной атакой является метод rainbow-таблиц, позволяющий практически взламывать хэши длиной до 64--128 бит.

Использование соли делает невозможной словарную атаку, так как после нахождения совпадения хэша в словаре злоумышленник вынужден будет подбирать соль, что вычислительно трудоемко или невозможно.

В рассмотренной ранее модели построения паролей в виде слогов с элементами небольшой модификации мы получили количество паролей около $2^{70}$ для 12-символьных паролей. Данный объем вычислений уже почти достижим. Следовательно, даже соль не защищает пароли от взлома, если у злоумышленника есть доступ к файлу с паролями или возможность неограниченных попыток аутентификации. Поэтому файлы с паролями дополнительно защищаются, а в системы аутентификации по паролю вводят ограничения на попытки неуспешной аутентификации.


\subsection[Unix]{Хранение паролей в Unix}

В ОС Unix пароль $m$ пользователя хранится в файле \texttt{/etc/shadow} в виде 128--512 битового хэша (SHA, MD5 и т.д.) или результатов шифрования (Blowfish и т.д.), вычисленного с солью $s$ из 2-8 или более символов (до 128--256 бит):
    \[ (s ~\|~ h_s), ~ \text{где} ~ h_s = h(s ~\|~ m). \]
Файл \texttt{/etc/shadow} доступен только привилегированным пользователям и процессам, что вносит дополнительную защиту.


\subsection[Windows]{Хранение паролей и аутентификация в \protect\\ Windows}

%[MS-NLMP]: NT LAN Manager (NTLM) Authentication Protocol Specification -- 09/25/2009, Rev. 11.0
%http://blogs.technet.com/authentication/archive/2006/04/07/ntlm-s-time-has-passed.aspx
%http://technet.microsoft.com/en-us/library/cc755284(WS.10).aspx -- Windows Authentication, Updated: February 7, 2008
%http://207.46.16.252/en-us/magazine/2006.08.securitywatch.aspx - The Most Misunderstood Windows Security Setting of All Time, Jesper Johansson
%http://en.wikipedia.org/wiki/NTLM
%http://www.windowsnetworking.com/nt/atips/atips92.shtml

ОС Windows, начиная с Vista, Server 2008, Windows 7, сохраняет пароли  в виде NT-хэша, который вычисляется как 128-битовый хэш MD4 от пароля в Unicode кодировке. NT-хэш не использует соль, поэтому применима словарная атака. На словарной атаке основаны программы поиска (взлома) паролей для Windows. Файл паролей называется SAM (Security Account Manager) в случае локальной аутентификации. Если пароли хранятся на сетевом сервере, то они хранятся в специальном файле, доступ к которому ограничен.

В последнем протоколе аутентификации NTLMv2\footnote{[MS-NLMP]: NT LAN Manager (NTLM) Authentication Protocol Specification, Rev. 11.}\index{NTLM, NTLMv2} пользователь для входа в свой компьютер аутентифицируется либо локально на компьютере, либо удаленным сервером, если учетная запись пользователя хранится на сервере. Пользователь с именем $user$ вводит пароль в программу-\emph{клиент}, которая, взаимодействуя с программой-\emph{сервером} (локальной или удаленной на сервере домена $domain$), аутентифицирует пользователя для входа в систему.
\begin{enumerate}
    \item Клиент $\rightarrow$ Сервер: запрос аутентификации.
    \item Клиент $\leftarrow$ Сервер: 64-битовая псевдослучайная одноразовая метка $n_s$.
    \item Вводимый пользователем пароль хэшируется в $\textrm{NThash}$ без соли. Клиент генерирует 64-битовую псевдослучайную одноразовую метку $n_c$, создает метку времени $ts$. Далее вычисляются 128-битовые коды аутентификации сообщений $\HMAC$ на хэш-функции MD5 с ключами $\textrm{NT-hash}$ и $\textrm{NTOWF}$:
        \[ \textrm{NThash} = \text{MD4}(\text{Unicode}(\text{пароль})), \]
        \[ \textrm{NTOWF} = \textrm{HMAC-MD5}_{\textrm{NThash}}(user, domain), \]
        %\[ \text{LMv2-response} = \text{HMAC-MD5}_{\text{NTLMv2-hash}}(n_c, n_s), \]
        \[ \textrm{NTLMv2-response} = \textrm{HMAC-MD5}_{\textrm{NTOWF}}(n_c, n_s, ts, domain). \]
    \item Клиент $\rightarrow$ Сервер: $(n_c, \textrm{NTLMv2-response})$. %LMv2-response,
    \item Сервер для указанных имен пользователя и домена извлекает из базы паролей требуемый NT-hash, производит аналогичные вычисления и сравнивает значения кодов аутентификации. Если они совпадают, аутентификация успешна.
\end{enumerate}

В случае аутентификации на локальном компьютере сравниваются значения $\textrm{NTOWF}$, вычисленное от пароля пользователя и извлеченное из файла паролей SAM.

Как видно, протокол аутентификации NTLMv2 обеспечивает одностороннюю аутентификацию пользователя серверу (или своему ПК).

При удаленной аутентификации по сети последние версии Windows используют протокол Kerberos, который обеспечивает взаимную аутентификацию, и только, если аутентификация по Kerberos не поддерживается клиентом или сервером, она происходит по NTLMv2.


\section[Аутентификация в интернет-сервисах]{Аутентификация в \protect\\ интернет-сервисах}

Взаимодействие браузера и веб-приложения происходит по протоколу HTTP, который не поддерживает состояния. Браузер, клиент, выполняет запросы к веб-серверу, а последний отсылает браузеру запрошенные файлы (HTML-страницы, изображения, стили, скрипты и вообще любые файлы). Для каждого запроса браузер указывает полный адрес (URI, uniform resource identifier). Самые часто используемые запросы: GET (получение файла с указанным URI), POST (отправка данных, введенных пользователем, и получение нового HTML файла веб-страницы), PUT (отправка файлов на указанный URI). При использовании javascript (AJAX) отправлять и получать данные можно без перезагрузки страницы, но взаимодействие происходит по стандартным HTTP-запросам.

\emph{Односторонняя} аутентификация пользователя интернет-сервисом состоит из двух фаз:
\begin{enumerate}
    \item первичная аутентификация по паролю при первом GET-запросе к сервису;
    \item вторичная аутентификация на основе токенов аутентификации, сгенерированных веб-приложением, для \emph{каждого} последующего запроса к сервису (GET, POST и т.д.). Каждый запрос к серверу должен быть аутентифицирован, так как протокол HTTP не поддерживает состояний и сеансов. Часто аутентификацию делают через cookie.
\end{enumerate}

\emph{Взаимная} аутентификация достигается \emph{двумя} разными односторонними аутентификациями:
\begin{enumerate}
    \item аутентификация пользователя интернет-сервисом по паролю и последующим сгенерированным токенам;
    \item аутентификация интернет-сервиса пользователем производится с помощью SSL-соединения, которое обеспечивает первичную аутентификацию на основе открытых ключей и вторичные аутентификации кодами аутентификации сообщения и шифрованием на сеансовых ключах, сгенерированных при первичной аутентификации. Первичная аутентификация веб-сервиса производится сертификатом X509, удостоверяющим принадлежность доменного имени URL веб-сервису, предотвращая фальсификацию веб-сервиса. Сертификат содержит ЭЦП \emph{общей доверенной} третьей стороны, вычисленной от значения открытых ключей, URL и других полей.
\end{enumerate}


\subsection{Первичная аутентификация по паролю}

Стандартная первичная аутентификация в современных интернет-сервисах происходит обычной передачей логина и пароля в открытом виде по сети. Если SSL-соединение не используется, логин и пароль могут быть перехвачены. Даже при использовании SSL-соединения веб-приложение имеет доступ к паролю в открытом виде.

Более защищенным, но мало используемым способом аутентификации является вычисление хэша от пароля $m$, соли $s$ и псевдослучайных одноразовых меток $n_1, n_2$ с помощью javascript в браузере и отсылка по сети только результата вычисления хэша.
\[ \begin{array}{ll}
    \text{Браузер} ~\rightarrow~ \text{Сервис:} & \text{HTTP GET-запрос} \\
    \text{Браузер} ~\leftarrow~ \text{Сервис:}  & s ~\|~ n_1 \\
    \text{Браузер} ~\rightarrow~ \text{Сервис:} & n_2 ~\|~ h( h(s ~\|~ m) ~\|~ n_1 ~\|~ n_2) \\
\end{array} \]

Если веб-приложение хранит хэш от пароля и соли $h(s ~\|~ m)$, то пароль не может быть перехвачен ни по сети, ни веб-приложением.

В массовых интернет-сервисах пароли часто хранятся в открытом виде на сервере, что не является хорошей практикой для обеспечения защиты персональных данных пользователей.


\input{openid}

\input{cookies_auth}

\chapter{Программные уязвимости}

\section[Контроль доступа в информационных системах]{Контроль доступа в \protect\\ информационных системах}
\selectlanguage{russian}

%http://www.acsac.org/2005/papers/Bell.pdf
%http://www.dranger.com/iwsec06_co.pdf
%http://csrc.nist.gov/groups/SNS/rbac/documents/design_implementation/Intro_role_based_access.htm
%http://en.wikipedia.org/wiki/Access_control#Computer_security
%http://en.wikipedia.org/wiki/Discretionary_access_control
%http://en.wikipedia.org/wiki/Mandatory_access_control
%http://en.wikipedia.org/wiki/Role-Based_Access_Control

В информационных системах контроль доступа вводится на \emph{действия} \emph{субъектов} над \emph{объектами}. В операционных системах под субъектами почти всегда понимаются процессы, под объектами -- процессы, разделяемая память, объекты файловой системы, порты ввода-вывода  и т.д., под действием -- чтение (файла или содержимого директории), запись (создание, добавление, изменение, удаление, переименование файла или директории) и исполнение (файла-программы). Система контроля доступа в информационной системе (операционной системе, базе данных и т.д.) определяет множество субъектов, объектов и действий.

Применение контроля доступа создается 1) \emph{аутентификацией} субъектов и объектов, 2) \emph{авторизацией} допустимости действия,  3) \emph{аудитом} (проверкой и хранением) ранее совершенных действий.

Различают три основных модели контроля доступа -- дискреционная (discretionary access control, DAC), мандатная (mandatory access control, MAC) и ролевая (role-based access control, RBAC) модели. Современные операционные системы используют \emph{комбинации} двух или трех моделей доступа, причем решения о доступе принимаются в порядке убывания приоритета: ролевая, мандатная, дискреционная модели.

В настоящее время в операционных системах использование и развитие контроля доступа происходит в том числе для повышения стабильности системы и устойчивости к ошибкам и уязвимостям, а не только для обеспечения секретности информации.


\subsection{Дискреционная модель}

Классическое определение дискреционной модели\index{контроль доступа!дискреционный} из так называемой Оранжевой книги 1985 г. (Trusted Computer System Evaluation Criteria, устаревший стандарт министерства обороны США 5200.28-STD) следующее -- средства ограничения доступа к объектам, основанные на сущности (identity) субъекта и/или группы, к которой они принадлежат. Субъект, имеющий определенный доступ к объекту, имеет возможность полностью или частично передать право доступа другому субъекту.

На практике дискреционная модель доступа предполагает, что для каждого объекта в системе определен субъект-владелец. Этот субъект может самостоятельно устанавливать необходимые, по его мнению, права доступа к любому из своих объектов для остальных субъектов, в том числе и для себя самого. Логически владелец объекта является владельцем информации, находящимся в этом объекте. При доступе некоторого субъекта к какому-либо объекту система контроля доступа лишь считывает установленные для объекта права доступа и сравнивает их с правами доступа субъекта. Кроме того, предполагается наличие в ОС некоторого выделенного субъекта -- администратора дискреционного управления доступом, который имеет привилегию устанавливать дискреционные права доступа для любых, даже ему не принадлежащих, объектов в системе.

Дискреционную модель реализуют почти все популярные ОС, в частности, Windows и Unix. У каждого субъекта (процесса пользователя или системы) и объекта (файла, другого процесса и т.д.) есть владелец, который может делегировать доступ другим субъектам, изменяя атрибуты на чтение, запись файлов для других пользователей и групп пользователей. Администратор системы имеет возможность назначить нового владельца и другие права доступа к объектам.


\subsection{Мандатная модель}

Классическое определение мандатной модели\index{контроль доступа!мандатный} из Оранжевой книги -- средства ограничения доступа к объектам, основанные на важности (секретности) информации, содержащейся в объектах, и обязательная авторизация действий субъектов для доступа к информации с присвоенным уровнем важности. Важность информации определяется уровнем доступа, приписываемым всем объектам и субъектам. Исторически мандатная модель определяла важность информации в виде иерархии, например, совершенно секретно (СС), секретно (С), конфиденциально (К) и рассекречено (Р). При этом верно следующее: СС $>$ C $>$ K $>$ P, то есть каждый уровень включает сам себя и все уровни, находящиеся ниже в иерархии.

Современное определение мандатной модели -- применение явно указанных правил доступа субъектов к объектам, определяемых только администратором системы. Сами субъекты (пользователи) не имеют возможности для изменения прав доступа. Правила доступа описаны матрицей, в которой столбцы соответствуют субъектам, строки -- объектам, а в ячейках -- допустимые действия субъекта над объектом. Матрица покрывает все пространство субъектов и объектов. Также определены правила наследования доступа для новых создаваемых объектов. В мандатной модели матрица может быть изменена только администратором системы.

Модель Белл-Ла Падулы (Bell-La Padula) использует два мандатных и одно дискреционное правила политики безопасности:
\begin{enumerate}
    \item Субъект с определенным уровнем секретности не может иметь доступ на \emph{чтение} объектов с более \emph{высоким} уровнем секретности (no read-up).
    \item Субъект с определенным уровнем секретности не может иметь доступ на \emph{запись} объектов с более \emph{низким} уровнем секретности (no write-down).
    \item Использование матрицы доступа субъектов к объектам для описания дискреционного доступа.
\end{enumerate}


\subsection{Ролевая модель}

Ролевая модель доступа основана на определении ролей в системе\index{контроль доступа!ролевой}. Понятие <<роль>> в этой модели -- это совокупность действий и обязанностей, связанных с определенным видом деятельности. Таким образом, достаточно указать тип доступа к объектам для определенной роли и определить группу субъектов, для которых она действует.
Одна и та же роль может использоваться несколькими различными субъектами (пользователями). В некоторых системах пользователю разрешается выполнять несколько ролей одновременно, в других есть ограничение на одну или несколько не противоречащих друг другу ролей в каждый момент времени.

Ролевая модель, в отличие от дискреционной и мандатной, позволяет реализовать разграничение полномочий пользователей, в частности, на системного администратора и офицера безопасности, что повышает защиту от человеческого фактора.


\input{os_access_controls}

\section{Виды программных уязвимостей}

В 1949 г. фон Нейман ввел понятие самовоспроизводящихся механических машин.
%Самовоспроизводящейся программой называется программа, создающая и распространяющая свои копии.

\textbf{Вирусом} называется самовоспроизводящаяся часть кода (подпрограмма)\index{вирус}, которая встраивается в носители (другие программы) для своего исполнения и распространения. Вирус не может исполняться и передаваться без своего носителя.

\textbf{Червем} называется самовоспроизводящаяся отдельная (под) программа\index{червь}, которая может исполняться и распространяться самостоятельно, не используя программу-носитель.

Первым сетевым вирусом считается вирус Creeper 1971 г., распространявшийся в сети Arpanet, предшественнике Интернета. Для его уничтожения был создан первый антивирус, Reaper, который находил и уничтожал Creeper.

Первый червь для Интернета, червь Морриса 1988 г., уже использовал \emph{смешанные} атаки для заражения UNIX машин:
% http://homes.cerias.purdue.edu/~spaf/tech-reps/823.pdf
%http://tools.ietf.org/html/rfc1135
\begin{itemize}
    \item переполнение буфера массива в стеке функции в результате копирования строки, выходящей за пределы отведенного буфера, функцией gets() и последующее исполнение заданного кода, содержащегося в копируемой строке;
    \item команду DEBUG удаленного сервиса sendmail для удаленного исполнения заданных команд,
    \item доступ к <<доверенным>> машинам без аутентификации;
    \item подбор паролей на основе: 1) модификаций логина, имени и фамилии пользователя, 2) словарной атаки на встроенном словаре из $4^{32}$ слов, 3) слов из орфографического словаря и дальнейшее подключение к другим машинам с аутентификацией по подобранным паролям.
\end{itemize}

Копируемая строка из 536 байтов для переполнения буфера содержала код запуска среды \texttt{/bin/sh}.
%\begin{verbatim}
%    pushl $0068732f             ’/sh\0’
%    pushl $6e69622f             ’/bin’
%    movl sp, r10
%    pushl $0
%    pushl $0
%    pushl r10
%    pushl $3
%    movl sp,ap
%    chmk $3b
%\end{verbatim}
%Код эквивалентен запуску \texttt{execve(``/bin/sh'', 0, 0)}.

\textbf{Программной уязвимостью}\index{программная уязвимость} называется свойство программы, позволяющее нарушить ее работу. Программные уязвимости могут приводить к отказу в обслуживании\index{отказ в обслуживании}\index{DoS-атака} (denial of service, DOS), утечке и изменению данных, появлению и распространению вирусов и червей.

Одной из распространенных атак для заражения персональных компьютеров является переполнение буфера массива в стеке функции. В интернет-сервисах наиболее распространенной программной уязвимостью в настоящее время является межсайтовый скриптинг (cross-site scripting, XSS).

Наиболее распространенные программные уязвимости можно разделить на классы:
\begin{enumerate}
    \item Переполнение буфера -- копирование в буфер данных большего размера, чем длина выделенного буфера. Буфером может быть контейнер текстовой строки, массив, динамически выделяемая память и т.д. Переполнение становится возможным вследствие либо отсутствия контроля над длиной копируемых данных, либо из-за ошибок в коде. Типичная ошибка -- разница в 1 байт между длиной буфера и данных при сравнении.
    \item Некорректная обработка (парсинг) данных, введенных пользователем, является причиной большинства программных уязвимостей в веб-приложениях. Под обработкой понимаются:
        \begin{enumerate}
            \item проверка на допустимые значения и тип (числовые поля не должны содержать строки и т.д.);
            \item фильтрация и экранирование специальных символов, имеющих значения в скриптовых языках или для декодирования из одной текстовой кодировки в другую. Примеры символов: \texttt{\textbackslash},  \texttt{\%}, \texttt{<}, \texttt{>}, \texttt{"},  \texttt{'};
            \item фильтрация ключевых слов языков разметки и скриптов. Примеры: \texttt{script}, \texttt{javaScript};
            \item декодирование различными кодировками при парсинге. Распространенный способ обхода системы контроля парсинга данных состоит в однократном или множественном последователельном кодировании текстовых данных в шестнадцатеричные кодировки \texttt{\%NN} ASCII и UTF-8. Например, браузер или веб-приложения производят $n$ -- кратные последовательные декодирования, в то время как система контроля делает $k$-кратное декодирование, $0 \leq k < n$,  и, следовательно, пропускает закодированные запрещенные символы и слова.
        \end{enumerate}
    \item Некорректное использование синтаксиса функций. Например, \texttt{printf(s)} может привести к уязвимости записи в указанный адрес памяти. Если злоумышленник вместо обычной текстовой строки введет в качестве \texttt{s = "текст некоторой длины\%n"}, то функция \texttt{printf()}, ожидающая первым аргументом строку формата \texttt{printf(fmt, \dots)}, обнаружив \texttt{\%n}, возьмет значения из ячеек памяти, следующих перед текстовой строкой (устройство стека функции описано далее), и запишет в адрес памяти, равный считанному значению, количество выведенных символов на печать функцией \texttt{printf()}.
\end{enumerate}


\input{stack_overflow}

\input{xss}

\input{sql-injections}

%\chapter{Послесловие}
%Это должно быть что-то в виде заключения, объяснения, почему именно эти темы выбраны, насколько актуален материал с теоретической и практической точки зрения.


\appendix

\chapter{Математическое приложение}
\label{chap:discrete-math}

Вычисления в криптосистемах с открытым ключом, как правило, выполняются в модульной арифметике, в группе или в поле. Далее мы рассмотрим базовые примеры групп и алгоритмов, используемых в криптосистемах с открытым ключом и алгоритме блокового шифрования AES.

\section*{Общие определения}

Выражением $\mod n$ обозначается вычисление остатка от деления произвольного целого числа на целое число $n$. В полиномиальной арифметике эта операция означает вычисление остатка от деления многочленов.
%далее будем обозначать целые числа или операции с целыми числами, взятыми \textbf{по модулю}\index{модуль} целого числа $n$ (остаток от целочисленного деления). Примеры выражений:
    \[ a\mod n, \]
    \[ (a + b) c\mod n. \]
Равенство
    \[ a = b \mod n \]
означает, что выражения $a$ и $b$ равны (говорят также <<сравнимы>>, <<эквивалентны>>) по модулю $n$.

Множество
    \[ \{ 0, 1, 2, 3,  \dots,  n-1 \mod n\} \]
состоит из $n$ элементов, где каждый элемент $i$ представляет все целые числа, сравнимые с $i$ по модулю $n$.

Наибольший общий делитель (НОД) двух чисел $a,b$ обозначается $\gcd(a,b)$ (the great common divisor).

Два числа $a,b$ называются взаимно простыми, если они не имеют общих делителей, $\gcd(a,b) = 1$.

Выражение $a \mid b$ означает, что $a$ делит $b$.

\section{Группы}
\selectlanguage{russian}

\subsection{Свойства групп}

\textbf{Группой}\index{группа} называется множество $\Gr$, в котором задана операция $\centerdot$ между двумя элементами группы и удовлетворяются аксиомы:
\begin{enumerate}
    \item замкнутость
        \[ \forall a,b \in \Gr: a \centerdot b = c \in \Gr; \]
    \item ассоциативность
        \[ \forall a,b,c \in \Gr: (a \centerdot b) \centerdot c = a \centerdot (b \centerdot c); \]
    \item существование единичного элемента
        \[ \exists ~ e \in \Gr: e\centerdot a = a \centerdot e = a; \]
    \item существование обратного элемента
        \[ \forall a \in \Gr ~ \exists ~ b \in \Gr: a \centerdot b = b \centerdot a = e. \]
\end{enumerate}
Если
    \[ \forall a,b \in \Gr: a \centerdot b = b \centerdot a, \]
то группа коммутативная.

Если операция в группе задана как умножение $\cdot$, то группа называется \textbf{мультипликативной}, $e = 1$, обратный элемент -- $a^{-1}$, возведение в степень $k$ -- $a^k$.

Если операция задана как сложение $+$, то группа называется \textbf{аддитивной}, $e = 0$, обратный элемент $-a$, сложение $k$ раз -- $ka$.

Подмножество группы, удовлетворяющее аксиомам группы, называется \textbf{подгруппой}\index{подгруппа}.

\textbf{Порядком} $|\Gr|$ \textbf{группы}\index{порядок группы} $\Gr$ называется число элементов в группе. Пусть группа мультипликативная. Для любого элемента $a \in \Gr$ выполняется $a^{|\Gr|} = 1$.

\textbf{Порядком элемента} $a$ называется минимальное число
    \[ ord(a): a^{ord(a)} = 1 \]
 Порядок элемента делит порядок группы:
    \[ ord(a) ~~|~~ |\Gr|. \]


\subsection{Циклические группы}

\textbf{Генератором} $g \in \Gr$ называется элемент, \emph{порождающий} всю группу\index{генератор группы}
    \[ \Gr = \{g, g^2, g^3,  \ldots,  g^{|\Gr|} = 1\}. \]
Группа, в которой существует генератор, называется \textbf{циклической}\index{группа!циклическая}.

Если конечная группа не циклическая, то в ней существуют циклические подгруппы, порожденные всеми элементами. Любой элемент $a$ группы порождает либо циклическую \emph{подгруппу}
    \[ \{ a, a^2, a^3,  \dots,  a^{ord(a)} = 1 \} \]
порядка $ord(a)$, если порядок элемента $ord(a) < |\Gr|$, или \emph{всю} группу
    \[ \Gr = \{ a, a^2, a^3,  \dots,  a^{|\Gr|} = 1 \}, \]
если порядок элемента равен порядку группы $ord(a) = |\Gr|$. Порядок любой подгруппы, как и порядок элемента, делит порядок всей группы.

Представим циклическую группу через генератор $g$ как
    \[ \Gr = \{g, g^2,  \ldots,  g^{|\Gr|} = 1\} \]
и  каждый элемент $g^i$  возведем в степени $1, 2,  \ldots,  |\Gr|$. Тогда
\begin{itemize}
    \item элементы $g^i$, для которых число $i$ взаимно просто с $|\Gr|$, породят снова всю группу
            \[ \Gr = \{ g^i, g^{2i}, g^{3i},  \dots,  g^{|\Gr| i} = 1 \}, \]
        так как степени $\{i, 2i, 3i, \dots |\Gr| i \}$ по модулю $|\Gr|$ образуют перестановку чисел $\{1, 2, 3, \dots, |\Gr|\}$; следовательно, $g^i$ -- тоже генератор, число таких чисел $i$ по определению функции Эйлера $\phi(|\Gr|)$ ($\phi(n)$ -- количество взаимно простых с $n$ целых чисел в диапазоне $[1,n-1]$);
    \item элементы $g^i$, для которых $i$ имеют общие делители
            \[ d_i = \gcd(i, |\Gr|) \neq 1 \]
        c $|\Gr|$, породят подгруппы
            \[ \{ g^i, g^{2i}, g^{3i},  \dots,  g^{\frac{i}{d_i} |\Gr|} = 1\}, \]
        так как степень последнего элемента кратна $|\Gr|$; следовательно, такие $g^i$ образуют циклические подгруппы порядка $d_i$.
\end{itemize}
%TODO Гашков, Болотов, Часовских "эллиптическая криптография" или "методы элл. кри-ии"

Из предыдущего утверждения следует, что число генераторов в циклической группе равно
    \[ \phi(|\Gr|). \]

Для проверки, является ли элемент генератором всей группы, требуется знать разложение порядка группы $|\Gr|$ на множители. Далее элемент возводится в степени, равные всем делителям порядка группы, и сравнивается с единичным элементом $e$. Если ни одна из степеней не равна $e$, то этот элемент является примитивным элементом или генератором группы. В противном случае элемент будет генератором какой-либо подгруппы.

Задача разложения числа на множители является трудной для вычисления. На сложности ее решения, например, основана криптосистема RSA. Поэтому при создании больших групп желательно знать, заранее, разложение порядка группы на множители для возможности выбора генератора.


\subsection{Группа $\Z_p^*$}

\textbf{Группой $\Z_p^*$} называется группа\index{группа!$\Z_p^*$}
    \[ \Z_p^* = \{1, 2,  \dots,  p-1 \mod p\}, \]
где $p$ -- простое число, операция в группе -- умножение $\ast$ по $\mod p$.

Группа $\Z_p^*$ -- \textbf{циклическая}, порядок
    \[ |\Z_p^*| = \phi(p) = p - 1. \]
Число генераторов в группе --
    \[ \phi(|\Z_p^*|) = \phi(p-1). \]

Из того, что $\Z_p^*$ -- группа, для простого $p$ и любого $a \in [2, p-1] \mod p$ следует \textbf{малая теорема Ферма}\index{теорема!малая теорема Ферма}:
    \[ a^{p-1} = 1 \mod p. \]
На малой теореме Ферма основаны многие тесты проверки числа на простоту.

\example
Рассмотрим группу $\Z_{19}^*$. Порядок группы -- 18. Делители: 2, 3, 6, 9. Является ли 12 генератором?
\[ \begin{array}{l}
    12^2 = -8 \mod 19, \\
    12^3 = -1 \mod 19, \\
    12^6 = 1 \mod 19, \\
\end{array} \]
12 -- генератор подгруппы 6 порядка. Является ли 13 генератором?
\[ \begin{array}{l}
    13^2 = -2 \mod 19, \\
    13^3 = -7 \mod 19, \\
    13^6 = -8 \mod 19, \\
    13^9 = -1 \mod 19, \\
    13^{18} = 1 \mod 19, \\
\end{array} \]
13 -- генератор всей группы.
\exampleend

\example
В таб. \ref{tab:Zp-sample} приведен пример группы $\Z_{13}^*$.  Число генераторов -- $\phi(12) = 4$. Подгруппы --
    \[ \Gr^{(2)}, \Gr^{(3)}, \Gr^{(4)}, \Gr^{(6)}, \]
верхний индекс обозначает порядок подгруппы.

\begin{table}[h!]
    \centering
    \caption {Генераторы и циклические подгруппы группы $\Gr=\Z_{13}^*$\label{tab:Zp-sample}}
    \resizebox{\textwidth}{!}{ \begin{tabular}{|c|p{0.66\textwidth}|c|}
        \hline
        Элемент & Порождаемая группа или подгруппа & Порядок \\
        \hline
        2 & $\Gr = \{ 2, 4,  8 = -5, -10 = 3, 6, 12 = -1, -2, -4, -8 = 5, 10 = -3, -6, -12 = 1 \}$ & 12, ген. \\
        3 & $\Gr^{(3)} = \{ 3, 9 = -4, -12 = 1 \}$ & 3 \\
        4 & $\Gr^{(6)} = \{ 4, 16 = 3, 12 = -1, -4, -3, -12 = 1 \}$ & 6 \\
        5 & $\Gr^{(4)} = \{ 5, 25 = -1, -5, 1 \}$ & 4 \\
        6 & $\Gr = \{6, 36 = -3, -5, -4, 2, -1, -6, 3, 5, 4, -2, -12 = 1 \}$ & 12, ген. \\
        7 = -6 & $\Gr = \{ -6, 36 = -3, 5, -4, -2, -1, 6, 3, -5, 4, 2, -12 = 1 \}$ & 12, ген. \\
        8 = -5 & $\Gr^{(4)} = \{ -5, 25 = -1, 5, 1 \}$ & 4 \\
        9 = -4 & $\Gr^{(3)} = \{ -4, 16 = 3, -12 = 1 \}$ & 3 \\
        10 = -3 & $\Gr^{(6)} = \{ -3, 9 = -4, 12 = -1, 3, -9 = 4, -12 = 1 \}$ & 6 \\
        11 = -2 & $\Gr = \{ -2, 4, 5, 3, -6, -1, 2, -4, -5, -3, 6, -12 = 1 \}$ & 12, ген. \\
        12 = -1 & $\Gr^{(2)} = \{ -1, 1 \}$ & 2 \\
        \hline
    \end{tabular} }
\end{table}
\exampleend


\subsection{Группа $\Z_n^*$}

\textbf{Функция Эйлера}\index{функция!Эйлера} $\phi(n)$ определяется как количество чисел, взаимно простых с $n$ , в интервале от 1 до $n-1$.

Если $n=p$ -- простое число, то
    \[ \phi(p) = p - 1, \]
    \[ \phi(p^k) = p^k - p^{k-1} = p^{k-1}(p - 1). \]
Если $n$ -- составное число и
    \[ n = \prod \limits_{i} p_i^{k_i} \]
разложено на простые множители $p_i$, то
    \[ \phi(n) = \prod \limits_{i} \phi(p_i^{k_i}) =  \prod \limits_{i} p_i^{k_i - 1}(p_i - 1). \]

\textbf{Группой $\Z_n^*$} называется группа\index{группа!$\Z_n^*$}
    \[ \Z_n^* = \left\{ \forall a \in \left\{ 1, 2,  \dots,  n-1 \mod n \right\} : \gcd(a,n) = 1 \right\}, \]
с операцией умножения $\ast$ по $\mod n$.

Порядок группы --
    \[ |\Z_n^*| = \phi(n). \]
Группа $\Z_p^*$ -- частный случай группы $\Z_n^*$.

Если $n$ \emph{составное} (не простое) число, то группа $\Z_n^*$ \textbf{нециклическая}.

Из того, что $\Z_n^*$ -- группа, для любых $a \neq 0, n > 1: \gcd(a,n) = 1$ следует \textbf{теорема Эйлера}\index{теорема!Эйлера}:
    \[ a^{\phi(n)} = 1 \mod n. \]

При возведении в степень, если $\gcd(a,n) = 1$, выполняется
    \[ a^b = a^{b \mod \phi(n)} \mod n. \]

\example
В табл. \ref{tab:Zn-sample} приведена нециклическая группа $\Z_{21}^*$ и ее циклические подгруппы
    \[ \Gr_1^{(2)}, \Gr_2^{(2)}, \Gr_3^{(2)}, \Gr_1^{(3)}, \Gr_1^{(6)}, \Gr_2^{(6)}, \Gr_3^{(6)}, \]
верхний индекс обозначает порядок подгруппы, нижний индекс нумерует различные подгруппы одного порядка.

\begin{table}[h!]
    \centering
    \caption{Циклические подгруппы нециклической группы $\Z_{21}^*$\label{tab:Zn-sample}}
    \begin{tabular}{|c|l|c|}
        \hline
        Элемент & Порождаемая циклическая подгруппа & Порядок \\
        \hline
        2  & $\Gr_1^{(6)} = \{ 2, 4, 8, 16, 11, 1 \}$ & 6 \\
        4  & $\Gr_1^{(3)} = \{ 4, 16, 1 \}$ & 3 \\
        5  & $\Gr_2^{(6)} = \{ 5, 4, 20, 16, 17, 1 \}$ & 6 \\
        8  & $\Gr_1^{(2)} = \{ 8, 1 \}$ & 2 \\
        10 & $\Gr_3^{(6)} = \{ 10, 16, 13, 4, 19, 1 \}$ & 6 \\
        11 & $\Gr_1^{(6)} = \{ 11, 16, 8, 4, 2, 1 \}$ & 6 \\
        13 & $\Gr_2^{(2)} = \{ 13, 1 \}$ & 2 \\
        16 & $\Gr_1^{(3)} = \{ 16, 4, 1 \}$ & 3 \\
        17 & $\Gr_2^{(6)} = \{ 17, 16, 20, 4, 5, 1 \}$ & 6 \\
        19 & $\Gr_3^{(6)} = \{ 19, 4, 13, 16, 10, 1 \}$ & 6 \\
        20 & $\Gr_3^{(2)} = \{ 20, 1 \}$ & 2 \\
        \hline
    \end{tabular}
\end{table}
\exampleend


\section{Конечные поля и операции в алгоритме AES}
\selectlanguage{russian}

В алгоритме блокового шифрования AES преобразования над байтами и битами осуществляются специальными математическими операциями. Биты и байты понимаются как элементы поля.

\subsection{Определение поля Галуа}

%Группой называется множество $\Gr$, в котором задана операция $\centerdot$ между двумя элементами группы и удовлетворяются аксиомы:
%\begin{enumerate}
%    \item Замкнутость -- $\forall a,b \in \Gr: a \centerdot b = c \in \Gr$.
%    \item Ассоциативность -- $\forall a,b,c \in \Gr: (a \centerdot b) \centerdot c = a \centerdot (b \centerdot c)$.
%    \item Существование единичного элемента -- $\exists ~ e \in \Gr: e\centerdot a = a \centerdot e = a$.
%    \item Существование обратного элемента -- $\forall a \in \Gr ~ \exists ~ b \in \Gr: a \centerdot b = b \centerdot a = e$.
%\end{enumerate}

\textbf{Полем} называется множество $\F$, для которого\index{поле}
\begin{itemize}
    \item заданы операции умножения <<$\cdot$>> и сложения <<$+$>>;
    \item выполняются аксиомы группы по сложению <<$+$>> для всего множества $\F$: \\
        1. для $a,b,c \in \F$ верно $a+b \in \F$, \\
        2. $(a+b)+c = a+(b+c)$, \\
        3. существует нулевой элемент -- ноль 0, $a+0=0+a=a$, \\
        4. существует единственный обратный элемент $-a$: \\
        \indent \indent \indent ~~~~~~ $a + (-a) = (-a) + a = 0$;
    \item выполняются аксиомы группы по умножению <<$\cdot$>> для множества $\{ \F \backslash 0 \}$, за исключением нуля: \\
        1. для $a,b,c \in \{ \F \backslash 0 \}$ верно \\
        \indent \indent \indent ~~~~~~ $a \cdot b \in \{ \F \backslash 0 \}$, \\
        \indent \indent \indent ~~~~~~ $(a \cdot b) \cdot c = a \cdot (b \cdot c)$, \\
        2. существует единичный элемент -- единица 1, ~ $a \cdot 1 = 1 \cdot a = a$, \\
        3. существует единственный обратный элемент $a^{-1}:$ \\
        \indent \indent \indent ~~~~~~ $a \cdot a^{-1} = a^{-1} \cdot a = 1$, \\
        к нулю 0 не существует обратного элемента и $a \cdot 0 = 0$;
%    \item Удовлетворяющее аксиомам группы по сложению и умножению, обратный элемент по умножению существует ко всем элементам кроме 0.
    \item операции сложения и умножения коммутативны
        \[ \begin{array}{l}
            a + b = b + a, \\
            a \cdot b = b \cdot a; \\
        \end{array} \]
    \item выполняется свойство дистрибутивности
        \[ a \cdot (b + c) = (a \cdot b) + (a \cdot c). \]
\end{itemize}

Примеры \emph{бесконечных} полей (с бесконечным числом элементов) -- поле рациональных чисел $\group{Q}$, поле вещественных чисел $\group{R}$, поле комплексных чисел $\group{C}$ с обычными операциями сложения и умножения.

В криптографии рассматриваются \emph{конечные} поля (с конечным числом элементов), называемые также \textbf{полями Галуа}.

Число элементов в любом конечном поле равно $p^n$, где $p$ -- простое число и $n$ -- натуральное число. Обозначения поля Галуа: $\GF{p}, \GF{p^n}, \F_p, \F_{p^n}$ (аббревиатура от Galois field). Все поля Галуа $\GF{p^n}$ изоморфны друг другу (существует взаимно однозначное отображение между полями, сохраняющее действие всех операций). Другими словами, существует только одно поле Галуа $\GF{p^n}$ для фиксированных $p, n$.

Приведем примеры конечных полей.

Двоичное поле $\GF{2}$ состоит из двух элементов:
    \[ \GF{2} = \{0, 1\} \]
с операцией $(\cdot)$ обычного умножения и сложения  $\oplus$ по модулю 2 (исключающее ИЛИ, XOR).

Поле
    \[ \GF{3} = \{0, 1, 2\} \]
состоит из 3-х элементов с операциями умножения $(\cdot \mod 3)$ и сложения $(+ \mod 3)$ по модулю.

Двоичное поле $\GF{2^n}$ строится \textbf{расширением} \emph{базового} поля $\GF{2}$. Элемент поля задается многочленом степени $n-1$ (или меньше) с коэффициентами из базового поля $\GF{2}$:
    \[ \alpha = \sum\limits_{i=0}^{n-1} a_i x^i, ~ a_i \in \GF{2}. \]
Сложение многочленов -- сложение коэффициентов при одинаковых степенях $x^i$ в поле $\GF{2}$, т.е. по $\text{XOR}$. Умножение многочленов в поле -- обычное умножение многочленов со сложением и умножением коэффициентов в поле $\GF{2}$ и дальнейшим приведением результата по модулю неприводимого многочлена $m(x)$ над полем $\GF{2}$.

Многочлен над базовым полем $\GF{p}$ называется \textbf{неприводимым}\index{многочлен!неприводимый}, если он не раскладывается на множители, и \textbf{приводимым}\index{многочлен!приводимый}, если раскладывается на множители.

Говорят, что неприводимый над базовым полем $\GF{p}$ многочлен $m(x)$ степени $n$ задает поле $\GF{p^n}$, если операция умножения в поле определена по модулю $m(x)$ (сложение определяется базовым полем, умножение -- многочленом $m(x)$). Количество элементов в поле определяется степенью $m(x)$) и равно $p^n$. Элементы поля есть остатки от деления на $m(x)$ и имеют степень не выше $n-1$.

Многочлен $g(x)$ называется \textbf{примитивным элементом}\index{многочлен!примитивный} (генератором) поля, если его степени порождают все ненулевые элементы, т.е. $\{ \GF{p^n} ~\backslash~ 0 \}$, заданное неприводимым многочленом $m(x)$, за исключением нуля:
    \[ \{ \GF{p^n} ~\backslash~ 0 \} = \{~ g(x), ~g^2(x), ~g^3(x),  \dots,  g^{p^n-1}(x) = 1 \mod m(x) ~\}. \]

Неприводимый многочлен $\mod m(x)$ называется  \textbf{примитивным}\index{многочлен!примитивный}, если $g(x)=x$.

\example
В табл. \ref{tab:irreducible-gf2} приведены примеры многочленов \emph{над полем} $\GF{2}$.
\begin{table}[h!]
    \centering
    \caption{Пример многочленов над полем $\GF{2}$\label{tab:irreducible-gf2}}
    \begin{tabular}{|c|c|c|}
        \hline
        Многочлен & \parbox{2.5cm}{Упрощенная запись} & Разложение \\
        \hline
        $'1' x + '0'$ & $x$ & неприводимый \\
        $'1' x + '1'$ & $x+1$ & неприводимый \\
        $'1' x^2 + '0' x + '0'$ & $x^2$ & $x \cdot x$ \\
        $'1' x^2 + '0'x + '1'$ & $x^2 + 1$ & $(x+1) \cdot (x+1)$ \\
        $'1' x^2 + '1' x + '0'$ & $x^2 + x$ & $x \cdot (x+1)$ \\
        $'1' x^2 + '1' x + '1'$ & $x^2 + x + 1$ & неприводимый \\
        $'1' x^3 + '0' x^2 + '0' x + '1'$ & $x^3 + 1$ & $(x+1) \cdot (x^2+x+1)$ \\
        \hline
    \end{tabular}
\end{table}
\exampleend


\subsection{Операции с байтами в AES}

Чтобы определить операции сложения и умножения двух байтов, введем сначала представление байта в виде многочлена степени 7 или меньше. Байт
    \[ a =( a_7, a_6, a_5, a_4, a_3, a_2, a_1, a_0) \]
преобразуется в многочлен $a(x)$ с коэффициентами 0 или 1 по правилу
    \[ a(x) = a_{7}x^{7}+a_{6}x^{6}+a_{5}x^{5}+a_{4}x^{4}+a_{3}x^{3}+a_{2}x^{2}+a_{1}x+a_{0}. \]

Далее байт трактуется как элемент конечного поля $\GF{2^8}$, заданного неприводимым многочленом
    \[ m(x) = x^{8}+x^{4}+x^{3}+x +1. \]

Произведение многочленов $a(x)$ и $b(x)$  по модулю многочлена $m(x)$  записывают как
    \[ c(x) = a(x) b(x) \mod m(x). \]
Остаток $c(x)$ представляет собой многочлен степени 7 или меньше. Его коэффициенты $(c_{7}, c_{6}, c_{5}, c_{4}, c_{3}, c_{2}, c_{1}, c_{0})$ образуют байт $c$, который и называется произведением байтов $a$ и $b$.

Сложение байтов осуществляется по $\oplus$ (исключающее ИЛИ), что является операцией сложения многочленов в двоичном поле.

\emph{Единичным} элементом поля является байт 00000001, или $\mathrm{'01'}$ в шестнадцатеричной записи. \emph{Нулевым} элементом поля является байт 00000000, или $\mathrm{'00'}$ в шестнадцатеричной записи. Одним из \emph{примитивных} элементов поля является байт (0 0 0 0 0 0 1 0), или $\mathrm{'02'}$ в шестнадцатеричной записи. Байты часто записывают в шестнадцатеричной форме, но при математических преобразованиях они должны интерпретироваться как элементы поля $\GF{2^8}$.

Для каждого ненулевого байта $a$ существует обратный байт $b$, такой, что их произведение является единичным байтом:
    \[ a b = 1 \mod m(x). \]
Обратный байт обозначается $b = a^{-1}$.

Для байта $a$, представленного многочленом $a(x)$, нахождение обратного байта $a^{-1}$ сводится к решению уравнения
    \[ m(x) d(x) + b(x) a(x) = 1. \]
Если такое решение найдено, то многочлен $b(x) \mod m(x)$ является представлением обратного байта $a^{-1}$. Обратный элемент (байт) может быть найден с помощью расширенного алгоритма Евклида для многочленов.

\example
Найти байт, обратный байту $a = \mathrm{'83'}$ в шестнадцатеричной записи. Так как $a(x) = x^{7} + x^{6} + 1$, то с помощью расширенного алгоритма Евклида находим
    \[ (x^{8} + x^{4} + x^{3} + x + 1) (x^{4} + x^{3} + x^{2} + x + 1) + (x^{7} + x^{6} + 1) (x^{5} + x^{3}) = 1. \]
Таким образом, обратный элемент поля или обратный байт $\mathrm{'83'}$ равен
    \[ x^{5} + x^{3}, ~ a^{-1} = \mathrm{'00101000'} = \mathrm{'28'}. \]
\exampleend

\example
В алгоритме блокового шифрования AES байты рассматриваются как элементы поля Галуа $\GF{2^8}$. Сложим байты $\mathrm{'57'}$ и $\mathrm{'83'}$. Представляя их многочленами, находим
    \[ (x^6 + x^4 + x^2 + x + 1) + (x^7 + x + 1) = x^7 + x^6 + x^4 + x^2, \]
или в двоичной записи --
    \[ 01010111 \oplus 10000011 = 11010100 = \mathrm{'D4'}. \]
Получили $\mathrm{'57'} + \mathrm{'83'} = \mathrm{'D4'}$.
\exampleend

\example
Выполним в поле $\GF{2^8}$, заданном неприводимым многочленом
    \[ m(x) = x^8 + x^4 + x^3 + x + 1 \]
(из алгоритма AES) операции с байтами: $\mathrm{'FA'} \cdot \mathrm{'A9'} + \mathrm{'E0'}$:
    \[ FA = 11111010, ~ A9 = 10101001, ~ E0 = 11100000, \]
    \[ (x^7 + x^6 + x^5 + x^4 + x^3  +x)(x^7 + x^5 + x^3 + 1) + (x^7 + x^6 + x^5) \mod m(x) = \]
    \[ = x^{14} + x^{13} + x^{10} + x^{8} + x^7 + x^3 + x \mod m(x) = \]
    \[ = (x^{14} + x^{13} + x^{10} + x^{8} + x^7 + x^3 + x) + x^6 \cdot m(x) \mod m(x) = \]
    \[ = x^{13} + x^9 + x^8 + x^6 + x^3 + x \mod m(x) = \]
    \[ = (x^{13} + x^9 + x^8 + x^6 + x^3 + x) + x^5 \cdot m(x) \mod m(x) = \]
    \[ = x^5 + x^3 + x \mod m(x) = \mathrm{'2A'}. \]
\exampleend


\subsection{Операции над вектором из байтов в AES}
%\subsection{Многочлены над полем в алгоритме AES}

Поле $\GF{2^{nk}}$ можно задать как расширение степени $nk$ базового поля $\GF{2}$:
    \[ \alpha \in \GF{2^{nk}}, \alpha = \sum\limits_{i=0}^{nk-1} a_i x^i, ~ a_i \in \GF{2} \]
с неприводимым многочленом $r(x)$ степени $nk$ над полем $\GF{2}$,
    \[ r(x) = \sum\limits_{i=0}^{nk} a_i x^i, ~ a_i \in \GF{2}, ~ a_{nk} = 1. \]

Поле $\GF{2^{nk}}$ можно задать и как расширение степени $k$ базового поля $\GF{2^n}$ :
    \[ \alpha \in \GF{(2^n}^k), \alpha = \sum\limits_{i=0}^{k-1} a_i x^i, ~ a_i \in \GF{2^n} \]
с неприводимым многочленом $r(x)$ степени $k$ над полем $\GF{2^n}$,
    \[ r(x) = \sum\limits_{i=0}^{k} a_i x^i, ~ a_i \in \GF{2^n}, ~ a_k = 1. \]

\example
В табл. \ref{tab:irreducible-gf8} приведены примеры приводимых и неприводимых многочленов над полем $\GF{2^8}$.
\begin{table}[h!]
    \centering
    \caption{Примеры многочленов над полем $\GF{2^8}$\label{tab:irreducible-gf8}}
    \begin{tabular}{|c|c|}
        \hline
        Многочлен & Разложение \\
        \hline
        $\mathrm{'01'} x + \mathrm{'00'}$ & неприводимый \\
        $\mathrm{'01'} x + \mathrm{'01'}$ & неприводимый \\
        $\mathrm{'01'} x + \mathrm{'A9'}$ & неприводимый \\
        $\mathrm{'01'} x^2 + \mathrm{'00'} x + \mathrm{'00'}$ & $(\mathrm{'01'} x) \cdot (\mathrm{'01'} x)$ \\
        $\mathrm{'1D'} x^2 + \mathrm{'AF'} x + \mathrm{'52'}$ & $(\mathrm{'41'} x + \mathrm{'0A'}) \cdot (\mathrm{'E3'} x + \mathrm{'5A'})$ \\
        $\mathrm{'01'} x^4 + \mathrm{'01'}$ & $(\mathrm{'01'} x + \mathrm{'01'})^4$ \\
        \hline
    \end{tabular}
\end{table}
\exampleend

В алгоритме AES вектор-столбец $\mathbf{a}$ состоит из четырех байтов $a_{0}, a_{1}, a_{2}, a_{3}$. Ему ставится в соответствие многочлен $\mathbf{a}(y)$ от переменной $y$ вида
    \[ \mathbf{a}(y) = a_{3}y^{3}+a_{2}y^{2}+a_{1}y+a_{0}, \]
причем коэффициенты многочлена (байты) интерпретируются как элементы поля $\GF{2^{8}}$. Это значит, что при сложении или умножении двух таких многочленов их коэффициенты складываются и перемножаются, как определено выше.

Многочлены $\mathbf{a}(y)$ и $\mathbf{b}(y)$ умножаются по модулю многочлена
    \[ \mathbf{M}(y)= \mathrm{'01'} y^4 + \mathrm{'01'} = y^4 + 1, ~ \mathrm{'01'} \in \GF{2^8}, \]
    \[ \mathbf{M}(y)= (\mathrm{'01'}, \mathrm{'00'},\mathrm{'00'}, \mathrm{'00'}, \mathrm{'01'}), \]
который \emph{не} является неприводимым над $\GF{2^8}$.
%Следовательно, многочлен $\mathbf{a}(y)$ задает многочлен третьей степени над полем $\GF{2^8}$, но не является элементом поля $\GF{2^{32}}$.

Операция умножения по модулю $\mathbf{M}(y)$  обозначается $\otimes$:
    \[ \mathbf{a}(y) ~ \mathbf{b}(y) \mod \mathbf{M}(y) ~\equiv~ \mathbf{a}(y) \otimes \mathbf{b}(y). \]

Операция <<Перемешивание столбца>> в шифровании AES состоит в умножении многочлена столбца на
    \[ \mathbf{c}(y) = (03, 01, 01, 02) = \mathrm{'03'} y^3 + \mathrm{'01'} y^2 + \mathrm{'01'} y + \mathrm{'02'} \]
по модулю $\mathbf{M}(y)$. Многочлен $\mathbf{c}(y)$ имеет обратный многочлен
    \[ \mathbf{d}(y) = \mathbf{c}^{-1}(y) \mod \mathbf{M}(y) = (\mathrm{0B}, \mathrm{0D}, \mathrm{09}, \mathrm{0E}) = \]
        \[ = \mathrm{'0B'} y^3 + \mathrm{'0D'} y^2 + \mathrm{'09'} y + \mathrm{'0E'}, \]
    \[ \mathbf{c}(y) \otimes \mathbf{d}(y) = (00, 00, 00, 01) = 1. \]
При расшифровании выполняется умножение на $\mathbf{d}(y)$ вместо $\mathbf{c}(y)$.

Так как
    \[ y^j = y^{j \mod 4} \mod y^4+1, \]
где коэффициенты из поля $\GF{2^8}$, то произведение многочленов
    \[ \mathbf{a}(y) = a_{3}y^{3}+ a_{2}y^{2} + a_{1}y + a_{0} \]
и
    \[ \mathbf{b}(y) = b_{3}y^{3} + b_{2}y^{2} + b_{1}y + b_{0}, \]
обозначаемое как
    \[ \mathbf{f}(y) = \mathbf{a}(y) \otimes \mathbf{b}(y) = f_{3}y^{3} + f_{2}y^{2} + f_{1}y + f_{0}, \]
содержит коэффициенты
\[
    \begin{array}{ccccccccc}
        f_{0} & = & a_{0}b_{0} & + & a_{3}b_{1} & + & a_{2}b_{2} & + & a_{1}b_{3}, \\
        f_{1} & = & a_{1}b_{0} & + & a_{0}b_{1} & + & a_{3}b_{2} & + & a_{2}b_{3}, \\
        f_{2} & = & a_{2}b_{0} & + & a_{1}b_{1} & + & a_{0}b_{2} & + & a_{3}b_{3}, \\
        f_{3} & = & a_{3}b_{0} & + & a_{2}b_{1} & + & a_{1}b_{2} & + &  a_{0}b_{3}.
    \end{array}.
\]

Эти соотношения можно переписать также в матричном виде:
\[
    \begin{array}{cccc}
        \left[ \begin{array}{c}
            f_{0} \\
            f_{1} \\
            f_{2} \\
            f_{3}
        \end{array} \right] &  = & \left[\begin{array}{cccc}
            a_{0} & a_{3} & a_{2} & a_{1} \\
            a_{1} & a_{0} & a_{3} & a_{2} \\
            a_{2} & a_{1} & a_{0} & a_{3} \\
            a_{3} & a_{2} & a_{1} & a_{0}
        \end{array}\right] & \left[\begin{array}{c}
            b_{0} \\
            b_{1} \\
            b_{2} \\
            b_{3}
        \end{array} \right]
    \end{array}.
\]

Умножение матриц производится в поле $\GF{2^8}$. Матричное представление полезно, если нужно умножать фиксированный вектор на несколько различных векторов.

\example
Вычислим для $\mathbf{a}(y) = (\mathrm{F2}, \mathrm{7E}, \mathrm{41}, \mathrm{0A})$ произведение $\mathbf{a}(y) \otimes \mathbf{c}(y)$:
\[
    \mathbf{c}(y) = (03, 01, 01, 02),
\] \[
    \mathbf{d}(y) = \mathbf{c}^{-1}(y) \mod \mathbf{M}(y) = (\mathrm{0B}, \mathrm{0D}, \mathrm{09}, \mathrm{0E}).
\] \[
    \mathbf{a}(y) \otimes \mathbf{c}(y) =
    \left[ \begin{array}{cccc}
        \mathrm{0A} & \mathrm{F2} & \mathrm{7E} & \mathrm{41} \\
        \mathrm{41} & \mathrm{0A} & \mathrm{F2} & \mathrm{7E} \\
        \mathrm{7E} & \mathrm{41} & \mathrm{0A} & \mathrm{F2} \\
        \mathrm{F2} & \mathrm{7E} & \mathrm{41} & \mathrm{0A} \\
    \end{array} \right] \cdot
    \left[ \begin{array}{c} \mathrm{02} \\ \mathrm{01} \\ \mathrm{01} \\ \mathrm{03} \end{array} \right] =
\] \[
    \left[ \begin{array}{ccccccc}
        \mathrm{0A} \cdot \mathrm{02} & \oplus & \mathrm{F2} & \oplus & \mathrm{7E} & \oplus & \mathrm{41} \cdot \mathrm{03} \\
        \mathrm{41} \cdot \mathrm{02} & \oplus & \mathrm{0A} & \oplus & \mathrm{F2} & \oplus & \mathrm{7E} \cdot \mathrm{03} \\
        \mathrm{7E} \cdot \mathrm{02} & \oplus & \mathrm{41} & \oplus & \mathrm{0A} & \oplus & \mathrm{F2} \cdot \mathrm{03} \\
        \mathrm{F2} \cdot \mathrm{02} & \oplus & \mathrm{7E} & \oplus & \mathrm{41} & \oplus & \mathrm{0A} \cdot \mathrm{03} \\
    \end{array} \right] =
    =\left[ \begin{array}{c} \mathrm{5B} \\ \mathrm{F8} \\ \mathrm{BA} \\ \mathrm{DE} \end{array} \right];
\] \[
    \begin{array}{l}
        \mathbf{a}(y) \otimes \mathbf{c}(y) = \mathbf{b}(y), \\
        \mathbf{b}(y) \otimes \mathbf{d}(y) = \mathbf{a}(y); \\
    \end{array}
\] \[
    \begin{array}{ccccc}
        (\mathrm{F2}, \mathrm{7E}, \mathrm{41}, \mathrm{0A}) & \otimes & (\mathrm{03}, \mathrm{01}, \mathrm{01}, \mathrm{02}) & = & (\mathrm{DE}, \mathrm{BA}, \mathrm{F8}, \mathrm{5B}), \\
        (\mathrm{DE}, \mathrm{BA}, \mathrm{F8}, \mathrm{5B}) & \otimes & (\mathrm{0B}, \mathrm{0D}, \mathrm{09}, \mathrm{0E}) & = & (\mathrm{F2}, \mathrm{7E}, \mathrm{41}, \mathrm{0A}). \\
    \end{array}
\]
\exampleend


\section{Модульная арифметика}
\selectlanguage{russian}

\subsection{Сложность модульных операций}

Криптосистемы с открытым ключом, как правило, построены в модульной арифметике с  длиной модуля от сотни до нескольких тысяч разрядов. Сложность алгоритмов оценивают как количество битовых операций в зависимости от длины. В табл. \ref{tab:mod-binary-complexity} приведены оценки (с точностью до порядка) сложности модульных операций\index{битовая сложность} для простых (или "школьных") алгоритмов вычислений. На самом деле, для реализации арифметики длинных чисел (сотни или тысячи двоичных разрядов) следует применять существенно более эффективные (более "хитрые") алгоритмы вычислений, использующие, например, специальный вид быстрого преобразования Фурье и другие приемы.

\begin{table}[h!]
    \centering
    \caption{Битовая сложность операций по модулю $n$ длиной $k= \log n$ бит\label{tab:mod-binary-complexity}}
    \begin{tabular}{| p{0.7\textwidth} | c |}
        \hline
        Операция, алгоритм & Сложность \\
        \hline
        1. $a \pm b \mod n$ & $O(k)$ \\
        2. $a \cdot b \mod n$ & $O(k^2)$ \\
        3. $\gcd(a, b)$, алгоритм Евклида & $O(k^2)$ \\
        4. $(a,b) \rightarrow (x,y,d) : ax + by = d = \gcd(a,b)$, расширенный алгоритм Евклида & $O(k^2)$ \\
        5. $a^{-1} \mod n$, расширенный алгоритм Евклида & $O(k^2)$ \\
        6. Китайская теорема об остатках & $O(k^2)$ \\
        7. $a^b \mod n$ & $O(k^3)$ \\
        \hline
    \end{tabular}

\end{table}


\subsection{Возведение в степень по модулю}

Метод называется <<возводи в квадрат и перемножай>>. Найдем $a^b \mod n$.
    \[ b = \sum_{i=0}^{k-1} b_i 2^i, \]
    \[ a^b = a^{\sum\limits_{i=0}^{k-1} b_i 2^i} = \prod_{i=0}^{k-1} (a^{{2^i} b_i} \mod n) \mod n. \]
Последовательно вычисляем квадраты
    \[ a_0 = a, ~ a_1 = a_0^2 \mod n, ~ a_2 = a_1^2 \mod n,  \ldots  \]
по модулю $n$ и перемножаем $a_i$, которым соответствует $b_i = 1$.  Число возведений в квадрат равно $k-1$ (если $b_{k-1} =1$), а число умножений меньше или равно $k-1$. Возведение в квадрат и умножение можно считать операцией с квадратичной битовой сложностью $O(k^2)$. Поэтому общая битовая сложность возведения в степень -- кубическая:
    \[ O(2(k-1)k^2) = O(k^3). \]

\example
\[ \begin{array}{l}
    8^{24} \mod 25 = 8^8 \cdot 8^{16} \mod 25, \\
    8^2 = 14, \\
    8^4 = -4, \\
    8^8 = 16, \\
    8^{16} = 6, \\
    8^{24} = 16 \cdot 6 = -4 \mod 25.
\end{array} \]
\exampleend


\input{euclidean_algorithm}

\input{chinese_remainder_theorem}

\section{(Псевдо) простые числа}

Функция $\pi(n)$ определяется как количество простых чисел из диапазона $[2, n]$.
Существует предел
    \[ \lim\limits_{n \rightarrow \infty}\frac{ \pi(n)}{ \frac{n}{\ln n}}=1. \]

Для $n \geq 17$ верно неравенство $\pi(n) > \frac{n}{\ln n}$.

Идея создания простых чисел состоит в случайном выборе числа и тестировании его на простоту.

Вероятность $P_k$ того, что случайное $k$-битовое число $n$ будет простым, равна
    \[ \lim\limits_{k \rightarrow \infty} P_k = \frac{1}{\ln n} = \frac{1}{k \ln 2}. \]

\example
    Вероятность того, что случайное 500-битовое число (включая четные числа) будет простым, примерно равна $\frac{1}{347}$, вероятность простоты случайного 2000-битового числа примерно равна -- $\frac{1}{1836}$.
\exampleend

\subsection{Тест Ферма}
\selectlanguage{russian}

Многие \emph{тесты на простоту} основаны на малой теореме Ферма: если $a$ и $n$ взаимно простые числа, то
    \[ a^{n-1} = 1 \mod n. \]

\textbf{Тестом Ферма}\index{тест!Ферма} на простоту числа $n$ по основанию $a$ называется процедура:
\begin{itemize}
    \item если для взаимно простых основания $a$ и модуля $n$ выполняется $a^{n-1} = 1 \mod n$, то $n$ \emph{может быть} простым,
    \item если $a^{n-1} \ne 1 \mod n$, то $n$ -- \emph{однозначно} составное.
\end{itemize}

Тесты есть \emph{детерминированные}, которые перебирают все $a$ до некоторой границы $a < A$, либо \emph{вероятностные}, которые проверяют тестом Ферма несколько псевдослучайных чисел $a$.

\textbf{Псевдопростым}\index{псевдопростое число} числом называется число, про которое не известно, является ли оно простым или нет, и удовлетворяющее вероятностному тесту на простоту с вероятностью ошибки теста меньше заданного $\epsilon$.

Оказывается, есть числа, которые удовлетворяют тесту Ферма для любого основания $a$. Числом Кармайкла\index{Кармайкла, число} называется составное число $n$, для которого тест Ферма выполняется для всех оснований $a$, взаимно простых с $n$. Первое число Кармайкла $561 = 3 \cdot 11 \cdot 17$. Чисел Кармайкла бесконечно много, но встречаются редко.

\example
Тест Ферма для числа Кармайкла $n = 561$ и взаимно простого с ним основания $a = 2$, приводится ниже:
\[
    n - 1 = 560 = 35 \cdot 2^4,
\] \[ \begin{array}{ll}
    2^{35} & =~ 2^1 \cdot 2^2 \cdot 2^{4 \cdot 0} \cdot 2^{8 \cdot 0} \cdot 2^{16 \cdot 0} \cdot 2^{32} ~= \\
        & =~ 2 \cdot 4 \cdot 16^0 \cdot 256^0 \cdot 460^0 \cdot 103 ~= \\
        & =~ 263 \mod 561, \\
\end{array} \] \[ \begin{array}{ll}
    2^{560} & =~ \left( 2^{35} \right)^{2^4} ~=~ 263^{2^4} ~= \\
        & =~ 166^{2^3} ~=~ 67^{2^2} ~=~ 1^{2} ~=~ 1 \mod 561.
\end{array} \]
\exampleend


\subsection{Вероятностный тест Миллера -- Рабина}
\selectlanguage{russian}

Улучшение теста Ферма основано на следующем утверждении: для простого $p$, из сравнения
    \[ a^2 = 1 \mod p, \]
    \[ (a-1)(a+1) = 0 \mod p \]
следует одно из двух
\[ \left[ \begin{array}{l}
     a = 1 \mod p, \\
     a = -1 \mod p. \\
\end{array} \right. \]

Для того, чтобы использовать это утверждение, представим нечетное число $n$  в виде произведения
    \[ n-1 = 2^s r, \]
где $r$ -- нечетное. Тогда получим
    \[ a^{n-1} = (a^r)^{2^s} \mod n. \]
Вначале вычислим $a_0 = a^r \mod n$ и последовательно возведем в квадрат $s$ раз:
    \[ a_i = a_{i-1}^2 \mod n, ~ i = 1, \dots, s. \]
Очевидно, что сравнение $a_s = 1 \mod n$ может выполниться в одном из двух случаев:
\[ \left[ \begin{array}{l}
    \text{либо}~ a_0 = a^r = 1 \mod n, \\
    \text{либо одно из чисел}~ a_i = -1 \mod n, ~ i \in [0, s-1]. \\
\end{array} \right. \]
Если одно из условий для данного $a$  выполняется, то  <<$n$ \emph{возможно} простое>>, если не выполняется, то -- <<$n$ \emph{однозначно} составное>>. Оказывается, что для любого нечетного составного числа $n$ по меньшей мере $\frac{3}{4}$ всех чисел $a$ являются \emph{свидетелями непростоты} числа $n$, то есть, для которых тест не будет пройден.

Вероятностный \textbf{тест Миллера--Рабина}\index{тест!Миллера-Рабина} состоит в проверке $t$ (псевдо)случайно выбранных чисел $a$. Если для всех $t$ чисел $a$ тест пройден, то $n$ называется псевдопростым\index{число!псевдопростое}, и вероятность того, что число $n$ не простое, имеет оценку
    \[ P_{error} < \left( \frac{1}{4} \right)^t. \]
Если для какого-то числа $a$ тест не пройден, то число $n$ точно составное.

Описание теста приведено в алгоритме $2$.

\begin{algorithm}[iht]
    \caption{Вероятностный тест Миллера--Рабина проверки числа на простоту.\label{miller-rabin}}
    \begin{algorithmic}
        \STATE Вход: нечетное $n>1$ для проверки на простоту и $t$ -- параметр надежности.
        \STATE Выход: \textsc{Составное} или \textsc{Псевдопростое}.
        \STATE $n - 1 = 2^s r, ~ r$ -- нечетное.
        \FOR{~$j = 1$ ~\textbf{to}~ $t$~}
            \STATE Выбрать (псевдо)случайное число $a \in [2, n-2]$.
            \IF{~$(a_0 = a^r ~\neq~ \pm 1 \mod n)$ ~\textbf{and} \\
            \indent ~~~~~~ $(\forall i \in [1, s-1]: ~ a_i = a_0^{2^i} ~\neq~ -1 \mod n)$~}
               \STATE \textbf{return} \textsc{Составное}.
           \ENDIF
        \ENDFOR
       \STATE \textbf{return} \textsc{Псевдопростое} с вероятностью ошибки $P_{error} < \left( \frac{1}{4} \right)^t$.
    \end{algorithmic}
\end{algorithm}

\example
В табл. \ref{tab-miller-rabin-sample} содержится пример теста Миллера--Рабина для $n = 169, ~ n-1 = 21 \cdot 2^3$.
\begin{table}[h!]
    \centering
    \caption{Пример теста Миллера--Рабина для $n = 169$ и четырех оснований $a$: 19, 22, 23, 2\label{tab-miller-rabin-sample}}
    \resizebox{\textwidth}{!}{ \begin{tabular}{||c|l|p{0.35\textwidth}||}
        \hline
        $a$ & $a_i \mod n$ & Вывод \\
        \hline \hline
        $19$ & $a_0 = a^r = 19^{21} = 70 \neq \pm 1 \mod 169$ & Возводим далее в квадрат \\
             & $a_1 = a_0^2 = -1 \mod 169$ & \textsc{Псевдопростое по основанию} $a=19$ \\
        \hline \hline
        $22$ & $a_0 = a^r = 22^{21} = 1 \mod 169$ & \textsc{Псевдопростое по основанию} $a=22$\\
        \hline \hline
        $23$ & $a_0 = a^r = 23^{21} = -1 \mod 169$ & \textsc{Псевдопростое по основанию} $a=23$\\
        \hline \hline
        $2$  & $a_0 = a^r = 2^{21} = 31 \neq \pm 1 \mod 169$ & Возводим далее в квадрат \\
             & $a_1 = a_0^2 = 116 \neq -1 \mod 169$ & Возводим далее в квадрат\\
             & $a_{s-1=2} = a_1^2 = 105 \neq -1 \mod 169$ & \textsc{Составное} \\
        \hline
    \end{tabular} }
\end{table}
\exampleend

Сложность алгоритма Миллера--Рабина для $k$-битового числа $n$ имеет порядок
    \[ O(t k^3) \]
двоичных операций.


\input{aks}

\subsection{Генерирование псевдопростых чисел}
\selectlanguage{russian}

Идея генерирования псевдопростого числа проста -- выбираем (псевдо)случайное число $n$ заданной битовой длины и проверяем его тестом на простоту. В среднем за $\ln n$ попыток встретится простое число. Если выбирать только нечетные числа, то среднее число попыток $\frac{\ln n}{2}$. На практике для проверки простоты числа обычно применяется тест Миллера--Рабина.

Пусть
    \[ \Delta_L = 2 \cdot 3 \cdot 5 \cdot ~\cdots~ \cdot p_L = \prod \limits_{p \leq p_L} p \]
произведение первых $L$ простых чисел. Из теоремы о распределении простых чисел следует
    \[ L \approx \frac{p_L}{\ln p_L}, ~~ p_L \approx L \ln L. \]
%TODO Что то из Чебышева
Для проверки на простоту случайное число $n_0$ нужной битовой длины выбирается взаимно простым с малыми простыми числами:
    \[ \gcd(n_0, \Delta_L) = 1. \]
Если $n_0$ не проходит тест Миллера--Рабина, то переходим к числу $n_1 = n_0 + \Delta_L$, которое также взаимно - простое с $\Delta_L$. И так далее, до тех пор пока тест не будет пройден  для некоторого
    \[ n_i = n_i + i \cdot \Delta_L. \]

Вероятность того, что случайное \textit{нечетное} число не будет иметь общих делителей с первыми $L$ простыми числами, равна
    \[ P(L) = \prod \limits_{3 \leq p \leq p_L} \left( 1 - \frac{1}{p} \right). \]
Используя приближение $1-x \leq e^{-x}$,
    \[ P(L) ~\lesssim~ e^{-\sum\limits_{3 \leq p \leq p_L} \frac{1}{p}} = e^{\frac{1}{2} ~ - \sum\limits_{p \leq p_L} \frac{1}{p}}. \]
Существует предел
    \[ \lim \limits_{n \rightarrow \infty} \left( \sum \limits_{p \leq n} \frac{1}{p} - \ln \ln n \right) = M, \]
называемый константой Мейсселя--Мертенса
    \[ M \approx 0.261497. \]
Упрощая уравнение, получаем
    \[ P(L) \approx e^{\frac{1}{2} - \ln \ln p_L - M} = \frac{e^{\frac{1}{2} - M}}{\ln(L \ln L)}. \]

Выбор числа, гарантированно не имеющего малых делителей (просеивание чисел), повышает шансы на то, что это число окажется простым. Например, для $L = 10^4$ вероятность, что 1024-битовое нечетное число
    \[ n \approx 2^{1024} \]
окажется простым, повышается в
    \[ \frac{1}{P(10^4)} \approx 10 \]
раз. В среднем каждое
    \[ \frac{\ln n}{2} \cdot \frac{1}{P(L)} \approx \frac{710}{2} \cdot 10 \approx 36 \]
нечетное число может быть простым вместо каждого $\frac{\ln n}{2} \approx 355$ числа (без просеивания), если нечетные числа выбирать без ограничений.

Средняя сложность генерирования $k$-битового псевдопростого числа имеет порядок
    \[ O \left( \frac{\ln n}{2} \cdot \frac{1}{P(L)} \cdot \left( t k^3 \right) \right) = O(t k^4). \]


\section{Группа точек эллиптической кривой над полем}
\selectlanguage{russian}

\subsection{Группы точек на эллиптических кривых}

Эллиптическая  кривая $E$ над полем вещественных чисел записывается в виде уравнения от двух переменных $x$ и $y$:

\begin{equation}
    E: ~ y^{2} = x^{3} + ax + b,
    \label{Wer}
\end{equation}

где $a,b \in \R$ -- вещественные числа. Эта форма представления эллиптической кривой называют формой Вейерштрасса.

На кривой определен инвариант

\begin{equation}
    J(E)=1728\frac{4a^{3} }{4a^{3} +27b^{2} }
    %\label{Inv}
\end{equation}

Пусть $x_{1} ,x_{2} ,x_{3} $ -- корни уравнения $x^3 + a x + b = 0$. Определим дискриминант $D$ в виде
    \[ D =(x_1 - x_2)^2 (x_1 - x_3)^2 (x_2 - x_3)^2 = - (4 a^3 + 27 b^2) \].

Рассмотрим различные значения дискриминанта $D$ и, соответствующие им  кривые, которые представлены на рисунках \ref{fig:elliptic-curve-1}, \ref{fig:elliptic-curve-2}, \ref{fig:elliptic-curve-3}.

\begin{enumerate}
    \item При $D>0$ эллиптическая кривая $y=y(x)$ состоит из двух частей (см. рис. \ref{fig:elliptic-curve-1}). Прямая, проходящая через точки $P(x_1, y_1)$ и $Q(x_2, y_2)$, обязательно пересечет вторую часть кривой в точке с координатами $(x_3, \widetilde{y}_3)$, отображением которой является точка $R(x_3, y_3)$, где $y_3 = - \widetilde{y}_3$. Любые точки на кривой при $D>0$ являются элементами группы по сложению.
        \begin{figure}[h!]
        	\centering
        	\includegraphics[width=0.5\textwidth]{pic/elliptic-curve-1}
            \caption{Эллиптическая кривая с дискриминантом $D>0$\label{fig:elliptic-curve-1}}
        \end{figure}
    \item Если $D=0$, то левая и правая части касаются в одной точке (см. рис. \ref{fig:elliptic-curve-2}). Эти кривые называются сингулярными и не рассматриваются.
        \begin{figure}[h!]
        	\centering
        	\includegraphics[width=0.5\textwidth]{pic/elliptic-curve-2}
            \caption{Эллиптическая кривая с дискриминантом $D = 0$\label{fig:elliptic-curve-2}}
        \end{figure}
    \item Если $D<0$, то записанное выше уравнение \ref{Wer} описывает одну кривую, представленную на рис. \ref{fig:elliptic-curve-3}.
        \begin{figure}[h!]
        	\centering
        	\includegraphics[width=0.5\textwidth]{pic/elliptic-curve-3}
            \caption{Эллиптическая кривая с дискриминантом $D < 0$\label{fig:elliptic-curve-3}}
        \end{figure}
\end{enumerate}

Рассмотрим операцию сложения точек на эллиптической кривой при $D>0$ (другие кривые не расматриваются).

Пусть точки $P(x_1, y_1)$ и $Q(x_2, y_2)$ принадлежат эллиптической кривой (рис. \ref{fig:elliptic-curve-1}). Определим операцию сложения точек
    \[ P + Q = R. \]

\begin{enumerate}
    \item Eсли $P \neq Q$, то точка $R$ определяется как отображение (инвертированная $y$-координата) точки, полученной пересечением эллиптической кривой и прямой $PQ$. Совместно решая уравнения кривой и прямой, можно найти координаты точки пересечения. Точка $R = (x_3, y_3)$ равна:
        \[ x_3 = \lambda^2 - x_1 - x_2, \]
        \[ y_3 = - y_1 + \lambda (x_1 - x_3), \]
        где
        \[ \lambda = \frac{y_2 - y_1}{x_2 - x_1} \]
        есть тангенс угла наклона между прямой, проходящей через точки $P$ и $Q$, и осью $x$.

        Теперь рассмотрим специальные случаи.
    \item Пусть точки совпадают: $P = Q$. Прямая $PQ$ превращается в касательную к кривой в точке $P$. Находим пересечение касательной с кривой, инвертируем $y$-координату полученной точки и это будет точка $P + P = R$. Тогда $\lambda$ -- тангенс угла между касательной, проведенной к эллиптической кривой в точке $P$, и осью $x$. Запишем уравнение касательной к эллиптической кривой в точке $(x,y)$ в виде
            \[ 2 y y' = 3 x^2 + a. \]
        Производная равна
            \[ y' = \frac{3 x^2 + a}{2 y} \]
        и
            \[ \lambda = \frac{3 x_1^2 + a}{2 y_1}. \]
        Координаты $R$ имеют прежний вид:
            \[ x_3 = \lambda^2 - x_1 - x_2, \]
            \[ y_3 = - y_1 + \lambda (x_1 - x_3), \]
    \item Пусть $P$ и $Q$ -- противоположные точки, то есть $P=(x,y)$ и $Q=(x, -y)$. Введем еще одну точку на бесконечности и обозначим ее $O$ (точка $O$ или точка 0 <<ноль>>, или альтернативное обозначение $\infty$). Результатом сложения двух противоположных точек определим точку $O$. Точка $Q$ в данном случае обозначается как $-P$:
        \[ P = (x,y), ~ -P = (x, -y), ~ P + (-P) = O. \]
    \item Пусть $P = (x, 0)$ лежит на оси $x$, тогда
        \[ -P = P, ~ P + P = O. \]
\end{enumerate}

Все точки эллиптической кривой, а также точка $O$ образуют группу $\E(\R)$ относительно введенной операции сложения, то есть выполняются законы группы\index{группа!точек эллиптической кривой}:
\begin{itemize}
    \item сумма точек $P + Q$ лежит на эллиптической кривой,
    \item существует нулевой элемент -- это точка $O$ на бесконечности:
        \[ \forall P \in \E(\R): ~ O + P = P \]
    \item для любой точки $P$ существует единственный обратный элемент $-P$:
        \[ P + (-P) = O; \]
    \item выполняется ассоциативный закон:
        \[ (P + Q) + F = P + (Q + F) = P + Q + F; \]
    \item выполняется коммутативный закон:
        \[ P + Q = Q + P. \]
\end{itemize}

Сложение точки с самой собой $d$ раз обозначим как умножение точки на число $d$:
    \[ \underbrace{P + P + \ldots + P}_{d \text{ раз}} = d P. \]


\subsection{Эллиптические кривые над конечным полем}

Эллиптические кривые можно строить не только над полем рациональных чисел, но и над другими полями. То есть координатами точек могут выступать не только числа, принадлежащие полю рациональных чисел $\R$, но также полю комплексных чисел $\mathbb{C}$ или конечному полю $\F$. В криптографии нашли своё применение эллиптические кривые именно над конечными полями.

Далее будем рассматривать эллиптические кривые над конечным полем, являющимся кольцом вычетов по модулю нечётного простого числа $p$ (дискриминант не равен 0):
    \[ E: ~ y^2 = x^3 + a x + b, \]
    \[ a, b, x, y \in \Z_p, \]
    \[ \Z_p = \{0, 1, 2,  \ldots,  p-1\},\]

Возможна также более компактная запись:

    \[ E: ~ y^2 = x^3 + a x + b \mod p\]

Точкой эллиптической кривой является пара чисел
    \[ (x,y): x, y \in \Z_p, \]
удовлетворяющая уравнению эллиптической кривой, определенной над конечным полем $\Z_p$.

Операцию сложения двух точек $P = (x_1, y_1)$ и $Q = (x_2, y_2)$ определим точно так же, как и в случае кривой над вещественным полем, описанным выше.

\begin{enumerate}
    \item Две точки $P = (x_1, y_1)$ и $Q = (x_2, y_2)$ эллиптической кривой, определенной над конечным полем $\Z_p$, складываются по правилу:
        \[
            P + Q = R \equiv (x_3, y_3),
        \] \[
            \left\{ \begin{array}{l}
                x_3 = \lambda^2 - x_1 - x_2 \mod p,\\
                y_3 = - y_1 + \lambda (x_1 - x_3) \mod p,\\
            \end{array} \right.
        \]
        где
        \[
            \lambda = \left\{ \begin{array}{l}
                \frac{y_2 - y_1}{x_2 - x_1} \mod p, ~ \text{ если } P \ne Q, \\
                \\
                \frac{3 x_1^2 + a}{2 y_1} \mod p, ~ \text{ если } P = Q. \\
            \end{array} \right.
        \]
    \item Сложение точки $P=(x,y)$ c противоположной точкой \\
        $(-P) = (x,-y)$ дает точку в бесконечности $O$:
        \[ P + (-P) = O, \]
        \[ (x_1, y_1) + (x_1, -y_1) = O, \]
        \[ (x_1, 0) + (x_1, 0) = O, \]
\end{enumerate}

Мы рассматриваем эллиптические кривые над конечным полем $\Z_p$, где $p > 3$ -- простое число, элементы $\Z_p$ -- целые числа $\{0, 1, 2,  \ldots,  p-1\}$, т.е. исследуем следующее уравнение двух переменных $x, y \in \Z_p$:
    \[ y^2 = x^3 + a x + b \mod p, \]
где $a, b \in \Z_p$ -- некоторые константы.

Как и в случае выше, множество точек над конечным полем $\Z_p$, удовлетворяющих уравнению эллиптической кривой, вместе с точкой в бесконечности $O$ образуют конечную группу $\E(\Z_p)$ относительно описанного закона сложения:\index{группа!точек эллиптической кривой}
    \[ \E(\Z_p) ~ \equiv~  O ~ \bigcup ~
        \left\{ (x, y) \in \Z_p \times \Z_p ~\Big|~ y^2 = x^3 + a x + b \mod p \right\}. \]

По теореме Хассе\index{теорема!Хассе} порядок группы точек $|\E(\Z_p)|$ оценивается как
    \[ (\sqrt{p}-1)^2 \leq |\E(\Z_p)| \leq (\sqrt{p}+1)^2, \]
или в другой записи
    \[ \Big| |\E(\Z_p)| - p - 1 \Big| \le 2 \sqrt{p}. \]

\subsection{Примеры группы точек}

\subsubsection{Пример 1}

Пусть эллиптическая кривая задана уравнением
    \[ E: ~ y^2 = x^3 + 1 \mod 7. \]
Найдем все решения этого уравнения, а также количество точек $|\E(\Z_p)|$ на этой эллиптической кривой. Для нахождения решений уравнения составим следующую таблицу:

\begin{center} \begin{tabular}{|c|c|c|c|c|c|c|c|}
    \hline
    $x$ & 0 & 1 & 2 & 3 & 4 & 5 & 6 \\
    \hline
    $y^2$ & 1 & 2 & 2 & 0 & 2 & 0 & 0 \\
    \hline
    $y_1$ & 1 & 3 & 3 & 0 & 3 & 0 & 0 \\
    \hline
    $y_2 = - y_1 \mod p$ & 6 & 4 & 4 &   & 4 &   &   \\
    \hline
\end{tabular} \end{center}

Выпишем все точки, принадлежащие данной эллиптической кривой $\E(\Z_p)$:
\[
    \begin{array}{cccc}
        P_1 = O, & P_2 = (0,1), & P_3 = (0,6), & P_4 = (1,3), \\
        P_5 = (1,4), & P_6 = (2,3), & P_7 = (2,4), & P_8 = (3,0), \\
        P_9 = (4,3), & P_{10} = (4,4), & P_{11} = (5,0), & P_{12} = (6,0). \\
    \end{array}
\]

Получили
    \[ |\E(\Z_p)| = 12. \]

Проверим выполнение неравенства Хассе:
    \[ \left| 12 - 7 - 1 \right| = 4 < 2 \sqrt{7}. \]
Следовательно, неравенство Хассе выполняется.

Введем следующие определения:
\begin{itemize}
    \item число $a$ называется квадратным вычетом, если уравнение $y^{2} =a \mod p$ имеет решение;
    \item число $a$ называется квадратным невычетом, если уравнение $y^{2} =a \mod p$ не имеет решения;
    \item минимальное число $s$ такое, что
        \[ \underbrace{P + P + \ldots + P}_{s} \equiv s P = O \]
        называется порядком точки.
\end{itemize}

%Теорема Лагранжа определяет порядок подгруппы.


\subsubsection{Пример 2}

Группа точек эллиптической кривой
    \[ y^2 = x^3 + 5 x + 6 \mod 17 \]
состоит из точек
\[ \begin{array}{ccccccc}
    \E(\Z_p) & =~ \Big\{ & (-8, \pm 7), & (-7, \pm 6), & (-6, \pm 7), &   & \\
             &           & (-5, \pm 3), & (-3, \pm 7), & (-1, 0),     & O & \Big\}. \\
\end{array} \]

Порядок группы:
    \[ |\E(\Z_p)| = 12. \]

Порядок группы точек по теореме Хассе:
    \[ (\sqrt{p}-1)^2 \leq |\E(\Z_p)| \leq (\sqrt{p}+1)^2, \]
    \[ 10 \leq 12 \leq 26. \]

Порядки возможных подгрупп: 2, 3, 4, 6 (все возможные делители порядка группы 12).

В табл. \ref{tab:ellipic-point-order-sample} найден порядок точки $P = (-8, 7)$ той же кривой
    \[ y^2 = x^3 + 5 x + 6 \mod 17. \]
Проверяются только степени точки, равные всем делителям порядка группы 12: 2, 3, 4, 6. Найденный порядок точки $(-8,7)$ равен 12, следовательно, она -- генератор всей группы.

\begin{table}[h!]
    \centering
    \caption{Пример нахождения порядка точки\label{tab:ellipic-point-order-sample}}
    \resizebox{\textwidth}{!}{ \begin{tabular}{|c|p{0.9\textwidth}|}
        \hline
        2 & $2 P = P + P = 2 \cdot (-8,7) = (-8,7) + (-8,7) = R$, \\
        & $\lambda = \frac{3 x_P^2 + a}{2y_P} = \frac{3 \cdot (-8)^2 + 5}{2 \cdot 7} = 8 \mod 17$, \\
        & $x_R = \lambda^2 - 2x_P = 8^2 - 2 \cdot (-8) = -5 \mod 17$, \\
        & $y_R = \lambda (x_P - x_R) - y_P = 8 \cdot ((-8) - (-5)) - 7 = 3 \mod 17$, \\
        & $R = 2P = (-5, 3)$ \\
        \hline
        3 & $3 P = 2 P + P = Q + P = R$, \\
        & $Q = 2P = (-5, 3)$, \\
        & $\lambda = \frac{y_Q - y_P}{x_Q - x_P} = \frac{3 - 7}{-5 - (-8)} = -7 \mod 17$, \\
        & $x_R = \lambda^2 - x_P - x_Q = (-7)^2 - (-8) - (-5) = -6 \mod 17$, \\
        & $y_R = \lambda (x_P - x_R) - y_P = -7 \cdot (-8 - (-6)) - 7 = 7 \mod 17$, \\
        & $R = 3P = (-6, 7)$ \\
        \hline
        4 & $4 P = 2 \cdot (2 P) = 2 \cdot (-5,3) = (-3, -7)$ \\
        \hline
        6 & $6 P = 2 P +  4 P = (-5,3) + (-3, -7) = (-1, 0)$ \\
        \hline
        12 & $12 P = 2 \cdot (6 P) = 2 \cdot (-1, 0) = O$ \\
        \hline
    \end{tabular} }
\end{table}

В табл. \ref{tab:elliptic-group-sample} найдены порядки точек и циклические подгруппы группы точек $\E(\Z_p)$ такой же эллиптической кривой
    \[ y^2 = x^3 + 5 x + 6 \mod 17. \]
Группа циклическая, число генераторов:
    \[ \phi(12) = 4. \]
Циклические подгруппы:
    \[ \Gr^{(2)}, ~ \Gr^{(3)}, ~ \Gr^{(4)}, ~ \Gr^{(6)}, \]
верхний индекс обозначает порядок подгруппы.

\begin{table}[h!]
    \centering
    \caption{Генераторы и циклические подгруппы группы точек эллиптической кривой\label{tab:elliptic-group-sample}}
    \resizebox{\textwidth}{!}{
    \begin{tabular}{|c|l|c|}
        \hline
        Элемент & Порождаемая группа или подгруппа & Порядок \\
        \hline
        $(-8,  \pm 7) $ & Вся группа $\E(\Z_p)$ & 12, генератор \\
        $(-7, \pm 6) $ & Вся группа $\E(\Z_p)$ & 12, генератор \\
        $(-6, \pm 7) $ & $\Gr^{(4)} ~=~ \big\{ ~ (-6, \pm 7), ~ (-1,0), ~ O ~ \big\}$ & 4 \\
        $(-5, \pm 3) $ & $\Gr^{(6)} ~=~ \big\{ ~ (-5, \pm 3), ~ (-3, \pm 7), ~ (-1,0), ~ O ~ \big\}$ & 6 \\
        $(-3, \pm 7) $ & $\Gr^{(3)} ~=~ \big\{ ~ (-3, \pm 7), ~ O ~ \big\}$ & 3\\
        $(-1, 0)     $ & $\Gr^{(2)} ~=~ \big\{ ~ (-1, 0), ~ O ~ \big\}$ & 2\\
        \hline
    \end{tabular}
    }
\end{table}


\section[Полиномиальные и экспоненциальные алгоритмы]{Полиномиальные и \\ экспоненциальные алгоритмы}

Данный раздел поясняет обоснованность стойкости криптосистем с открытым ключом и имеет лишь косвенное отношение к дискретной математике.

Машина Тьюринга (МТ) (модель, представляющая любой вычислительный алгоритм) состоит из следующих частей:
\begin{itemize}
    \item неограниченной ленты, разделенной на клетки; в каждой клетке содержится символ из конечного алфавита, содержащего пустой символ blank; если символ ранее не был записан на ленту, то он считается blank;
    \item печатающей головки, которая может считать, записать символ $a_i$ и передвинуть ленту на 1 клетку влево-вправо $d_k$;
    \item конечной таблицы действий
    \[ (q_i, a_j) \rightarrow (q_{i1}, a_{j1}, d_k), \]
где $q$ -- состояние машины.
\end{itemize}

Если таблица переходов однозначна, то машина Тьюринга\index{машина Тьюринга} называется детерминированной. \textbf{Детерминированная} машина Тьюринга может \emph{имитировать} любую существующую детерминированную ЭВМ. Если таблица переходов не однозначна, то есть, $(q_i, a_j)$ может переходить по нескольким правилам, то машина \textbf{недетерминированная}. \emph{Квантовый компьютер} является примером недетерминированной машины Тьюринга.

Класс задач $\set{P}$ -- задачи, которые могут быть решены за \emph{полиномиальное} время\index{задача!полиномиальная} на \emph{детерминированной} машине Тьюринга. Пример полиномиальной сложности (количество битовых операций)
    \[ O(k^{\textrm{const}}), \]
где $k$ -- длина входных параметров алгоритма. Операция возведения в степень в модульной арифметике $a^b \mod n$ имеет кубическую сложность $O(k^3)$, где $k$ -- двоичная длина чисел $a,b,n$.

Класс задач $\set{NP}$ -- обобщение класса $\set{P} \subseteq \set{NP}$, включает задачи, которые могут быть решены за \emph{полиномиальное} время на \emph{недетерминированной} машине Тьюринга. Пример сложности задач из $\set{NP}$ -- экспоненциальная сложность\index{задача!экспоненциальная}
    \[ O(\textrm{const}^k). \]
Описанный алгоритм Гельфонда (в разделе криптостойкости системы Эль-Гамаля) решения задачи дискретного логарифма по нахождению $x$ для заданных $g \mod p$ и $a = g^x \mod p$ имеет сложность $O(e^{k/2})$, где $k$ -- двоичная длина чисел.

В криптографии полиномиальные $\set{P}$ алгоритмы считаются \emph{легкими и вычислимыми} на ЭВМ, которые являются детерминированными машинами Тьюринга. Неполиномиальные (экспоненциальные) $\set{NP}$ алгоритмы считаются \emph{трудными и невычислимыми} на ЭВМ, так как из-за экспоненциального роста сложности всегда можно выбрать такой параметр $k$, что время вычисления станет сравнимым с возрастом Вселенной.

Задача факторизации числа, задача дискретного логарифмирования в группе считаются $\set{NP}$-задачами.

Класс $\set{NP}$-полных задач -- подмножество задач из $\set{NP}$, для которых не известен полиномиальный алгоритм для детерминированной машины Тьюринга, и все задачи могут быть сведены друг к другу за полиномиальное время на \emph{детерминированной} машине Тьюринга. Например, задача об укладке рюкзака является $\set{NP}$-полной.

Стойкость криптосистем с \emph{открытым} ключом, как правило, основана на $\set{NP}$ или $\set{NP}$-полных задачах:
\begin{enumerate}
    \item RSA -- $\set{NP}$-задача факторизации (строго говоря, на трудности извлечения корня степени $e$ по модулю $n$).
    \item Криптосистемы типа Эль-Гамаля -- $\set{NP}$-задача дискретного логарифмирования.
\end{enumerate}

\emph{Нерешенной} проблемой является доказательство неравенства
    \[ \set{P} \neq \set{NP}. \]
Именно на гипотезе о том, что для для некоторых задач не существует полиномиальных алгоритмов, и основана стойкость криптосистем с открытым ключом.

\section{Метод индекса совпадений}
\selectlanguage{russian}
\label{chap:coincide-index}

Приведем теоретическое обоснование метода индекса совпадений. Пусть алфавит имеет размер $A$. Перенумеруем его буквы числами от $1$ до $A$. Пусть заданы вероятности появления каждой буквы
    \[ \mathcal{P} = \left\{ {p_1 ,p_2 ,  \ldots , p_A } \right\}. \]
В простейшей модели языка предполагается, что тексты состоят из последовательности букв, порождаемых источником независимо друг от друга с известным распределением $\mathcal{P}$.

Найдем индекс совпадений для различных предположений относительно распределений букв последовательности. Сначала рассмотрим случай, когда вероятности всех букв одинаковы. Пусть
    \[ \mathbf{X} = \left[ X_1, X_2, \dots, X_L \right] \]
случайный текст с распределением
    \[ \mathcal{P}_1 = \left\{ p_{11}, p_{12}, \dots, p_{1A} \right\}. \]
Найдем индекс совпадений
    \[ I_c(\mathcal{P}_1), \]
то есть, вероятность того, что в случайно выбранной паре позиций находятся одинаковые буквы.

Для пары позиций $(k,j)$ найдем условную вероятность $P \left( X_k  = X_j \mid (k,j) \right)$:
    \[ P \left( X_k  = X_j \mid (k,j) \right) ~=~ \sum\limits_{i=1}^A p_{1i}^2 ~\equiv~ k_{p_1}. \]
Эта вероятность не зависит от выбора пары позиций $(k,j)$.

Так как число различных пар равно $\frac{L(L - 1)}{2}$, то вероятность случайного выбора пары $(k,j)$  равна
    \[ P_{(K,J)} (k,j) = \frac{2}{L(L - 1)}. \]
Следовательно,
\[
    I(\mathcal{P}_1) ~= \sum \limits_{1 \leq k < j \leq L} P_{(K,J)}(k,j) ~\cdot~ P(X_k  = X_j \mid (k,j)) =
\] \[
    = \sum \limits_{1 \leq k < j \leq L} \frac{2}{L(L - 1)} k_{p_1} = k_{p_1}.
\]

Найдем теперь аналогичную вероятность $I\left( {\mathcal{P}_1 ,\mathcal{P}_2 } \right)$  для случая, когда последовательность независимых случайных букв может быть представлена в виде
\[
\mathbf{X} = \left[ {\begin{array}{*{20}c}
   {X_1 ,}  \\
   {Y_1 ,}  \\
 \end{array} \begin{array}{*{20}c}
   {X_2 ,}  \\
   {Y_2 ,}  \\
 \end{array} \begin{array}{*{20}c}
   { \ldots ,}  \\
   { \ldots ,}  \\
 \end{array} \begin{array}{*{20}c}
   {X_{L/2} }  \\
   {Y_{L/2} }  \\
 \end{array} } \right],
\]
где одинаково распределенные случайные буквы в первой строке имеют распределение
    \[ \mathcal{P}_1  = \left\{ {p_{11} ,p_{12} ,  \ldots , p_{1A} } \right\}, \]
а одинаково распределенные случайные буквы во второй строке имеют другое распределение
    \[ \mathcal{P}_2  = \left\{ {p_{21} ,p_{22} ,  \ldots , p_{2A} } \right\}. \]
В этом случае сумму по всем парам мы разделяем на три суммы: по парам внутри позиций первой строки, по парам внутри позиций второй строки и по парам, в которых первая позиция берется из первой строки, а вторая –- из второй:
{ \small
\[
    I(\mathcal{P}_1, \mathcal{P}_2) =
        \frac{2}{L(L - 1)} \cdot \left(
        \sum \limits_{1 \leq k < j \leq L/2} P( X_k  = X_j \mid ( k,j )) ~~ + \right.
\] \[
        \left. + \sum\limits_{1 \leq k < j \leq L/2} P(Y_k  = Y_j \mid (k,j)) ~~+~~
            \sum\limits_{k=1}^{L/2} \sum\limits_{j=1}^{L/2} {P(X_k = Y_j \mid (k,j))} \right) =
\] \[
    = \frac{2}{L(L - 1)} \left( \frac{1}{2} \frac{L}{2} \left( \frac{L}{2} - 1 \right) k_{p_1} +
        \frac{1}{2} \frac{L}{2} \left( \frac{L}{2} - 1 \right) k_{p_2} +
        \left( \frac{L}{2} \right)^2 \sum \limits_{i = 1}^A p_{1,i} p_{2,i} \right) =
\] \[
    = \frac{2}{L(L - 1)} \left( \frac{1}{2} \frac{L}{2} \left( \frac{L}{2} - 1 \right) k_{p_1} +
        \frac{1}{2} \frac{L}{2} \left( \frac{L}{2} - 1 \right) k_{p_2} +
        \left( \frac{L}{2} \right)^2 k_{p_1, p_2} \right),
\] }
где обозначено
    \[ k_{p_1, p_2}  = \sum\limits_{i=1}^A p_{1,i} p_{2,i}. \]


В общем случае рассмотрим последовательность, представленную в виде матрицы, состоящей из $m$  строк и $\frac{L}{m}$ столбцов, где
\[
{\mathbf X} = \left[ {\begin{array}{*{20}c}
   {X_1 } & {X_2 } &  \ldots  & {X_{L/m} }  \\
   {Y_1 } & {Y_2 } &  \ldots  & {Y_{L/m} }  \\
    \vdots  &  \vdots  &  \vdots  &  \vdots   \\
   {Z_1 } & {Z_2 } &  \ldots  & {Z_{L/m} }  \\
\end{array}} \right].
\]
Считаем, что одинаково распределенные случайные буквы в первой строке имеют распределение
    \[ P_1  = \left\{ {p_{11} ,p_{12} ,  \ldots , p_{1A} } \right\}, \]
одинаково распределенные случайные буквы во второй строке имеют распределение
    \[ P_2  = \left\{ {p_{21} ,p_{22} ,  \ldots , p_{2A} } \right\} \]
и т.д., одинаково распределенные случайные буквы $m$-й строки имеют распределение
    \[ P_m  = \left\{ {p_{m1},p_{m2} ,  \ldots , p_{mA} } \right\}. \]

Для вычисления вероятности того, что в случайно выбранной паре позиций  будут одинаковые буквы, выполним суммирование по различным парам внутри строк и по парам между различными строками. Аналогично предыдущему случаю получим
{ \small
\[
    I(\mathcal{P}_1, \mathcal{P}_2, \ldots, \mathcal{P}_m ) ~=
\] \[
    =~ \frac{2}{L(L - 1)} \left( \frac{1}{2} \frac{L}{m} \left( \frac{L}{m} - 1 \right) k_{p_1} ~+~
        \frac{1}{2} \frac{L}{m} \left( \frac{L}{m} - 1 \right) k_{p_2} ~+ \right.
\] \[
        +~ \dots ~+~ \left. \frac{1}{2} \frac{L}{m} \left( \frac{L}{m} - 1 \right) k_{p_m} \right) ~+
\] \[
       +~ \frac{2}{L(L - 1)} \left( \left( \frac{L}{m} \right)^2 k_{p_1, p_2} +
         \left( \frac{L}{m} \right)^2 k_{p_1, p_3} + \dots +
        \left( \frac{L}{m} \right)^2 k_{p_{m - 1}, p_m } \right).
\] }
Первая фигурная скобка содержит  $m$ слагаемых, вторая -- $ \frac{m(m-1)}{2}$ слагаемых. Полагая
    \[ k_{p_1} = k_{p_2} = \dots = k_{p_m} = k_p, \]
    \[ k_{p_i p_j } = k_r = \frac{1}{A}, ~ i \ne j, \]
получим после несложных выкладок
    \[ m = \frac{k_p  - k_r}{I - k_r  + \frac{k_p  - I}{L}}. \]


%\chapter{Задачи и упражнения}
%
%К \textbf{примерам 1, 2, 3} \textbf{упражнение 1}.  Указать способ расшифрования для легального получателя шифротекста и указать способ дешифрования для криптоаналитика, не знающего ключа.
%
%\textbf{Упражнение 2}. Пусть $M_1, M_2, M_3,\ldots,M_s$ -- набор перестановок. Показать, что существует единственная перестановка $M=M_1,M_2,M_3,\ldots M_s$.
%
%\textbf{Упражнение 3}. Вскрыть одиночную ячейку Фейстеля. Для этого задать конкретную функцию $F(K,R)$ и по конкретным значениям $L_{1}$ и $R_{1}$ найти $K$.
%
%\textbf{Упражнение 4}. Разделим последовательность на блоки, каждый из которых содержит 2 бита.
%
%\[\begin{array}{cc} {z_{1} } & {z_{2} } \end{array}|\begin{array}{cc} {z_{3} } & {z_{4} } \end{array}| \ldots |\begin{array}{cc} {z_{N} } & {z_{N+1} } \end{array}\]
%Блок может принимать значения $z_{1} z_{2} =\begin{array}{c} {11} \\ {10} \\ {01} \\ {00} \end{array}$
%Преобразуем последовательность символов:
% если $z_{1} z_{2} =11$ или $z_{1} z_{2} =00$, то пара выбрасывается;
%если $[z_{1} z_{2} =10$, то записываем новый символ $u=1$; если
%$z_{1} z_{2} =01$, то записываем новый символ $u=0$.
%Получаем новую двоичную последовательность.
%
%Показать, что вероятностное распределение символов в новой последовательности является равномерным.
%
%\textbf{Упражнение 5}.Предположим, что криптоаналитик знает, что период генерируемой $M$ -последовательности равен $T=2^{L} -1$. Пусть ему известен часть последовательности длины, меньшей периода: $T_{1}<2^{L} -1$.
% При каком значении $T_{1}$  криптоаналитик может найти многочлен обратной связи.
%
%\textbf{Упражнение 6}. Ответить на вопрос: <<Как подделать ЭЦП, не зная секретного ключа?>>
%
%\textbf{Упражнение 7}. При помощи формул Виета найти дискриминант многочлена, представляющего эллиптическую кривую.

\newpage
\printindex

\newpage
%\input{bibliography}
\bibliographystyle{gost705}
\bibliography{bibliography}

\end{document}
